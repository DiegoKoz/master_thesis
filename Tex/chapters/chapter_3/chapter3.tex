%\documentclass[../../tesis.tex]{subfiles}
\documentclass[class=article, crop=false]{standalone}
\usepackage[subpreambles=true]{standalone}
\usepackage{import}
\graphicspath{{images/}}


\usepackage{booktabs}

%
\usepackage{amssymb}
\usepackage{amsmath}
%\usepackage{natbib}
\setcounter{tocdepth}{3}
\usepackage{graphicx}
%\graphicspath{ {Graficos/} }
\usepackage{subfigure}
\usepackage{gensymb}
\usepackage{authblk}
\usepackage{url}
\usepackage[utf8]{inputenc}

\usepackage[spanish]{babel}
\selectlanguage{spanish}
%\usepackage[style=authoryear]{biblatex}


\usepackage{nameref}


\usepackage{tikz}
\usepackage{tkz-berge}
\usetikzlibrary{positioning}



\newcommand{\bipgraph}[2]{%
	\begin{tikzpicture}[every node/.style={circle,draw}]
	\foreach \xitem in {1,2,...,#1}
	{%
		% first set
		\node at (0,\xitem) (a\xitem) {};
		% second set
		\node at (2,\xitem) (b\xitem) {};   
	}%
	
	% connections
	\foreach \x [count=\xi] in {#2}
	{% 
		\foreach \tritem in \x % <-- Here no braces to make it a foreach list also not \xi but \x
		\draw(a\xi) -- (b\tritem);
	}
	\end{tikzpicture}  
}



\begin{document}
	
	\section{introducción}
	
	En el presente capítulo\footnote{Parte del trabajo del presente capítulo se realizó en el marco  del proyecto de Investigación Científica y Tecnológica (PICT) 1085-2016. 'Abordando la restricción externa en América Latina a partir de la integración regional: integración productiva, cooperación Sur-Sur y financiamiento para el desarrollo'.} se analiza la estructura del comercio internacional a partir de la estructura de la canasta exportadora de los países. El rol que ocupan los países en el mercado mundial esta profundamente determinado por cómo se insertan en las cadenas globales de valor, y por ende en los tipos de mercancías que producen para el mercado mundial \citep{coe2004globalizing, gereffi2005governance,gereffi1994organization}. Por ello, luego de analizar la centralidad de los países desde la topología de la red, resulta de interés interrogarse hacer de los determinantes de dicha posición, a partir de su estructura productiva. 
	
	Es amplia la literatura que estudia el comercio internacional a partir del nivel producto \citep{balassa1965trade,lall2000technological,lall2006sophistication,haveman2004alternative} y en particular existe una extensa bibliografía reciente en la cual se analiza este fenómeno a paritr de construcción de un grafo bipartito entre países y productos \citep{10.1371/journal.pone.0197575,straka2017grand,ferreira2016topology,caldarelli2012network}. Destaca el trabajo de \cite{Hidalgo2009a}\citep{Hidalgo2009, Hidalgo2007}, donde se define el espacio de producto en términos de la complejidad de la producción y las influencias que ello tiene sobre el desarrollo de los países.
	
	El presente capítulo propone, luego de un análisis exploratorio de la información a partir de treemaps, dos modelos alternativos para abordar el fenómeno. En primer lugar, inspirado en el modelo propuesto para detección de tópicos en minería de textos \citep{Blei2003}, se propone un modelo generativo bayesiano para detección de dimensiones latentes en el espacio de productos, que permite construir un nomenclador alternativo a los convencionales, construido de forma automática basado en los datos. Luego, con dichas dimensiones latentes, \textit{componentes}, se analiza la participación de las mismas en la canasta exportadora de los países. En segundo lugar, fuertemente inspirado en la propuesta metodológica de \cite{Hidalgo2009}, se realizan diferentes técnicas de clustering y detección de comunidades, tanto sobre la proyección del grafo bipartito entre países y productos, como a partir de una matriz de distancias del espacio de productos.
	
	
	
	\section{Análisis exploratorio de datos}
	
	El análisis exploratorio de la información desagregada a nivel producto implica un mayor nivel de complejidad, dado que se incorpora una nueva dimensión, de alta cardinalidad, al estudio. Dado que la información a nivel de comercio agregado entre países ya fue realizada en el capítulo previo, en este análisis exploratorio se realizará foco en comprender la composición de las canastas exportadoras e importadoras de los países.
	
	Dado que los nomencladores de productos en su nivel desagregado implican una alta cardinalidad, el estudio de su distribución resulta inabarcable. Por ello, tradicionalmente el análisis económico recurre a diferentes niveles de agregación. Para realizar el análisis exploratorio de esta sección recurrimos al nomenclador elaborado por  \cite{molinari2016especializacion} a partir del concepto de \textit{cadenas productivas}. Este nomenclador tiene dos niveles de agregación: Cadenas y Subcadenas (ver Apéndice).
	
		
	La figura \ref{fig:treemaps_global} muestra la distribución de las exportaciones según Cadenas, Subcadenas y Usos, para el promedio mundial durante el período 1996-2016. En \ref{fig:treemaps_global_1} se puede ver que las dos cadenas más importantes son las de Bienes de Capital y Otras Manufacturas, dentro de las cuales las subcadenas que destacan son los equipos eléctricos y de transporte. 	Les sigue la cadena de insumos difundidos, donde destacan los metales y químicos. También son importantes las subcadenas de autopartes, autos, y dentro de la cadena de combustibles, el gas y petróleo. En la figura \ref{fig:treemaps_global_2} las cadenas se subdividen según el uso que se le da a los productos: Productos Primarios, Bienes de Capital, etc (ver Apéndice). Vale mencionar que la cadena Bienes de Capital no incluye exclusivamente productos de este tipo, tal como se observa en la figura, dado que las cadenas hacen referencia a las Cadenas de valor \citep{humphrey2000governance}, es decir a la rama de la producción a la que pertenecen los productos del final de la cadena. Allí podemos observar que en la cadena de Otras manufacturas destacan los productos semiterminados y bienes de consumo, mientras que en las cadena de Bienes de capital destacan su homónimo y las partes y componentes. A su vez en la cadena de insumos difundidos destacan los productos semiterminados ampliamente, mientras que en la industria automotriz se exportan mayoritariamente partes y componentes. 
	
	\begin{figure}
		\centering
		\subfigure[Cadenas y Subcadenas]{\label{fig:treemaps_global_1}\includegraphics[width=.65\linewidth]{treemap_cadsubcad.png}}
		\subfigure[Cadenas y Usos]{\label{fig:treemaps_global_2}\includegraphics[width=.65\linewidth]{treemap_usos.png}}
		\caption{Treemaps por tipos de productos. Exportaciones. 1996-2016. Total mundial}
		\label{fig:treemaps_global}
	\end{figure}
	
	
	
	La figura \ref{fig:treemaps_global_paises} muestra la distribución de las exportaciones e importaciones según continente y país exportador, para el promedio 1996-2016. Allí destaca el mayor volumen de exportaciones que importaciones de Asia, y el mayor volumen de importaciones que exportaciones de Estados Unidos, siguiendo las mismas conclusiones que el capítulo 2. 
	
	
	\begin{figure}
		\centering
		\includegraphics[width=.65\textwidth]{treemap_paises}
		\caption{Treemaps por Continentes y países. 1996-2016. Total mundial}
				\label{fig:treemaps_global_paises}
	\end{figure}
	
	
	Sin embargo, la riqueza en el análisis a nivel producto se encuentra en comprar la distinta composición de la balanza comercial de los países. Ya sea la diferencia entre la composición de sus exportaciones respecto de sus importaciones, como de estos elementos respecto de otros países o regiones. Dado que dicho análisis implica la comparación de una multiplicidad de treemaps, se decidió elaborar una herramienta interactiva para el análisis. La misma fue elaborada utilizando la librería \textit{shiny} \citep{Chang2018} y se encuentra disponible en \hyperlink{https://treemaps.shinyapps.io/treemaps/}{https://treemaps.shinyapps.io/treemaps/}. Dados los límites del servidor y el interés de estudiar la herramienta para un caso específico, el análisis se centra en el caso latinoamericano, tanto para el subcontinente en conjunto como para los países que lo integran. El objeto de análisis es estudiar las posibilidades de integración regional de la producción, y para ello analizar el comportamiento diferencial de las canastas exportadoras e importadoras de los países latinoamericanos cuando comercian entre sí respecto de su comercio con el resto del mundo. A su vez, dada la importancia del comercio de esta región con China, se decidió dividir la información del resto del mundo exceptuando a China y este país de forma individual.
	
	En la figura \ref{fig:treemaps_sudamerica_cadsubcad} se pueden observar los treemaps de cadenas y subcadenas para 2016 del total de los países sudamericanos, según su destino. En la figura \ref{fig:treemaps_sudamerica_cadsubcad_1} se ve como la canasta exportadora de los países latinoamericanos varía según si su destino se encuentra dentro o fuera de la región, y en particular si son exportaciones hacia China. En particular, el comercio intra-regional tiene un componente importante de bienes de capital, y del sector automotriz, ya sean autos terminados o autopartes. Por su parte, en el comercio con el resto del mundo estos componentes cumplen un rol secundario, mientras que se destacan las cadenas agroindustriales, de combustibles e insumos industriales. Respecto del resto del mundo, el comercio con China resalta por las exportaciones de la subcadena de biocombusitbles, cereales y oleaginosas y Metales. 
	Las importaciones de estos mismos países se pueden observar en la figura \ref{fig:treemaps_sudamerica_cadsubcad_2}. Naturalmente los treemaps de las exportaciones e importaciones son muy similares, ya que la única diferencia es el país que registra la operación. Sin embargo, en las importaciones del resto del mundo destacan las cadenas automotriz y de bienes de capital. A su vez, resulta de interés el cambio de composición de las cadenas: Mientras que las exportaciones de insumos industriales hacia el resto del mundo son mayoritariamente metales, en las importaciones destacan los productos químicos. Por su parte, mientras que en la cadena de otras manufacturas se exportan hacia el resto del mundo metales y piedras preciosas, se importan dentro de esta cadena equipos eléctricos. A su vez, del resto del mundo se importan medicamentos, aunque este rubro no aparece en el treemap de las exportaciones. Cabe destacar que la región exporta e importa petróleo hacia el resto del mundo, aunque el comercio intrarregional de este producto es menor. Existe, por lo tanto, una potencialidad de integración comercial dentro de la región para este producto en particular. 
	El comercio con China esta particularmente orientado a la venta de materias primas y la compra de productos industriales. Destacan especialmente las exportaciones de cereales y oleaginosas, biocombustibles y metales; mientras que se importan equipos eléctricos, bienes de capital, calzados y químicos. 
	

\begin{figure}
	\centering
	\subfigure[Exportaciones]{\label{fig:treemaps_sudamerica_cadsubcad_1}\includegraphics[width=.75\linewidth]{treemaps_cadsubcad2016}}
	\subfigure[Importaciones]{\label{fig:treemaps_sudamerica_cadsubcad_2}\includegraphics[width=.75\linewidth]{treemaps_cadsubcad2016_impos}}
	\caption{Treemaps de Cadenas y Subcadenas.2016. Total Latinoamérica}
	\label{fig:treemaps_sudamerica_cadsubcad}
\end{figure}


La figura \ref{fig:treemaps_sudamerica_usos} muestra los treemaps exclusivamente según los usos de los productos, para el total de la región, según su destino, en el año 2016. En \ref{fig:treemaps_sudamerica_usos_1} se puede observar las exportaciones. Los usos de los productos exportados varían fuertemente según su destino. Los productos semiterminados constituyen la mayoría de las exportaciones tanto hacia el interior de la región, como con el resto del mundo. Sin embargo, mientras que los bienes de consumo resultan de particular importancia dentro de la región, los productos primarios lo son respecto del resto del mundo. El comercio con china destaca esta tendencia, donde la mayoría de las exportaciones provienen de productos primarios. En \ref{fig:treemaps_sudamerica_usos_2} se observa la distribución de las importaciones según su uso. Allí se ve que las importaciones de productos semiterminados desde el resto del mundo supera a las exportaciones, si se lo compara con \ref{fig:treemaps_sudamerica_usos_1}. A su vez, prácticamente no se exportan productos primarios, pero sí partes y componentes, y bienes de capital. El comercio con China se encuentra equitativamente distribuido entre bienes de capital, productos semiterminados, de consumo y partes y componentes. Es notorio que no se importa prácticamente productos primarios, aunque estos constituyen la base de las exportaciones a dicho país. 

\begin{figure}
	\centering
	\subfigure[Exportaciones]{\label{fig:treemaps_sudamerica_usos_1}\includegraphics[width=.75\linewidth]{treemaps_usos2016}}
	\subfigure[Importaciones]{\label{fig:treemaps_sudamerica_usos_2}\includegraphics[width=.75\linewidth]{treemaps_usos2016_impos}}
	\caption{Treemaps de Cadenas y Usos.2016. Total Latinoamérica}
	\label{fig:treemaps_sudamerica_usos}
\end{figure}




En la figura \ref{fig:treemaps_sudamerica_destino} se observa la distribución de los socios comerciales de Sudamérica en el 2016. Esto es, los países destino de las exportaciones y origen de las importaciones. En \ref{fig:treemaps_sudamerica_destino_1} se excluye al resto del mundo y China, para observar la distribución interna del comercio. Naturalmente Brasil es el mayor socio comercial interno, para ambos tipos de flujo comercial, aunque su importancia es mucho mayor como exportador que como importador, dado que es el origen del 40\% de las importaciones de los demás países, y el destino del 25\% de las exportaciones. 
En la figura \ref{fig:treemaps_sudamerica_destino_2} se puede observar el mismo gráfico, pero incluyendo al resto del mundo. Allí se ve que tanto las exportaciones como importaciones hacia el resto del mundo exceptuando a China constituyen más de un 60\% del comercio, y China exclusivamente representa casi un 20\% del comercio de la región, superando ampliamente a Brasil. 

\begin{figure}
	\centering
	\subfigure[Excluyendo Resto del Mundo]{\label{fig:treemaps_sudamerica_destino_1}\includegraphics[width=.65\linewidth]{treemaps_paises2016}}
	\subfigure[Incluyendo Resto del mundo]{\label{fig:treemaps_sudamerica_destino_2}\includegraphics[width=.65\linewidth]{treemaps_paises_rdm2016}}
	\caption{Treemaps según socio comercial. 2016. Total Latinoamérica}
	\label{fig:treemaps_sudamerica_destino}
\end{figure}
	

De las figuras \ref{fig:treemaps_sudamerica_cadsubcad}, \ref{fig:treemaps_sudamerica_usos} y \ref{fig:treemaps_sudamerica_destino} se desprende que la región de Sudamérica tiene una baja integración regional en tanto una pequeña proporción de su comercio se realiza entre países de la región. A su vez, mientras el comercio intrarregional se realiza fundamentalmente sobre productos de alta complejidad y valor agregado, el comercio con el resto del mundo se encuentra desbalanceado: mientras las exportaciones se concentran en productos primarios de baja complejidad, aquellos productos que llegan desde el resto del mundo tienen un mayor grado de tecnificación. Estas conclusiones son particularmente válidas para el comercio con China, que constituye una quinta parte del comercio total de la región. Vale mencionar que la importancia de dicho país ah crecido de forma sostenida en el período analizado, siendo que en 1996 representaba menos de 2\% del comercio \footnote{ver \hyperlink{https://treemaps.shinyapps.io/treemaps/}{https://treemaps.shinyapps.io/treemaps/}}

\section{Metodología}

Del análisis exploratorio se desprende el potencial de la información desagregada a nivel producto para caracterizar la inserción en el mercado mundial de un país o región. Sin embargo, dada la alta cardinalidad de la información y la múltiples dimensiones de estudio, resulta dificultoso su estudio de forma general, sin hacer eje en un determinado país, o sin agregar la información según un nomenclador. En particular, el uso de agrupaciones jerárquicas de los productos resulta esencial para el análisis de resultados. La elaboración de los mismos constituye una extensa tarea por parte de expertos en las temáticas sectoriales de los diferentes tipos de productos, y el nomenclador resultante esta fuertemente determinado por los objetivos con que sera utilizado. Es por ello que en la presente sección se propone una elaboración alternativa de niveles jerárquicos de agrupamiento de productos. 

\info{agregar grafo bipartito}

	
\subsection{Latent Dirichlet Allocation Models}
	
El objetivo de la presente sección es elaborar un agrupamiento automático de los productos, basado en la información disponible. Este problema puede ser concebido desde dos puntos de vista: por un lado, se puede pensar como un problema de \textit{clustering} donde lo que se busca es crear grupos de productos con similares características. Por otro lado, es un problema de reducción de dimensionalidad, donde lo que se busca es encontrar un espacio de menor dimensión al que existen originalmente los datos. Esto sería posible dado que existe la posibilidad de explotar las similitudes y diferencias entre los distintos productos. 

Sin embargo, las técnicas de clustering tradicionales encuentran problemas en espacios de alta dimensionalidad \citep{aggarwal2001surprising}. A su vez, siguiendo el trabajo de \cite{molinari2016especializacion}, se entiende que los grupos no deberían ser excluyentes, dado que un mismo producto puede ser utilizado en diferentes formas, como producto intermedio o final. En este sentido, el problema se puede especificar como de \textit{clustering difuso}.
	
En concreto, la dimensionalidad del problema se puede pensar como el siguiente espacio:

$$
\mathcal{R}^{N*P*Y*2}
$$
Es decir la interacción de $N$ países, $P$ productos, $Y$ años, tanto para las exportaciones como las importaciones.
La propuesta es utilizar la técnica propuesta por \cite{blei2003latent} conocida como \textit{Latent Dirichlet Allocation Models} o \textit{Topic Modeling}. En su versión original, está técnica se propone como una forma de encontrar los tópicos presentes en un corpus, y la distribución de dichos tópicos sobre cada texto. Este problema es análogo al que se busca en el presente trabajo: Allí se busca una dimensión latente de $k$ tópicos, embebidos en un diccionario de alta dimensionalidad (las palabras presentes en el corpus), que se distribuyen a lo largo de los textos que componen dicho corpus. En el presente problema, se busca una dimensión latente de $k$ \textit{componentes}, embebidos en un nomenclador de alta dimensionalidad, que se distribuyen a lo largo de los países. Esta técnica a su vez puede ser pensada como un problema de clustering difuso, en tanto cada palabra (en el contexto original) puede pertenecer a más de un tópico.

A continuación se realizará una descripción del modelo propuesto por \cite{blei2003latent}, adaptado al presente dominio. 
	
\subsubsection{Definiciones}

\begin{itemize}
\item
Un \textbf{\emph{producto}} es la \emph{unidad básica discreta de los
	datos}, se define como un ítem de un nomenclador (SITC). Se representa
como un vector unitario. Definimos el superíndice \(i\) del vector
como el i-ésimo producto del nomenclador y el i-ésimo elemento del
vector. El V-ésimo producto del nomenclador es el vector \(w\), tal
que \(w^v\)=1 y \(w^u\)=0, \(u\neq v\)
\item
Un \textbf{\emph{país\&año}} es una secuencia de \textbf{N} productos,
definido como \(W= (w_1, w_2, ..., w_N)\)
\item
Nuestro \textbf{corpus} es la colección de \textbf{M} países, definido
como \(D = (d_1, d_2,..., d_M)\)
\item
Un \textbf{\emph{componente}} es una dimensión latente sobre el
corpus, y suponemos una cantidad fija \emph{k} de los mismos.
\item
Nuestro objetivo es obtener:
\end{itemize}

\begin{enumerate}
\def\labelenumi{\arabic{enumi}.}
\item
Una distribución de componentes sobre cada país\&año.
\item
Una distribución de los productos sobre los componentes.
\end{enumerate}


\subsubsection{Proceso generativo e inferencial}

Intuitivamente, suponemos el siguiente proceso generativo de datos:

\begin{itemize}
\item
Para cada país del corpus, imaginamos que las exportaciones surgen de
un proceso de dos etapas:

\begin{itemize}
	\item
	Elegimos aleatoriamente una distribución sobre los componentes
	\item
	Para cada dólar exportado:
	
	\begin{itemize}
		\item
		Elegimos aleatoriamente el componente al que pertence, dada la
		distribución definida en el paso anterior
		\item
		Elegimos aleatoriamente un producto de la distribución
		correspondiente a dicho componente
	\end{itemize}
\end{itemize}
\end{itemize}

\begin{enumerate}
\def\labelenumi{\arabic{enumi}.}
\item
Para cada Componente $k \in \{1,2,... K\}$
\end{enumerate}

\begin{itemize}
\item
Generar una distribución sobre los componentes
$\beta \sim Dir_v(\eta)$ con $\eta \in \mathcal{R}_{>0}$ un
parámetro fijo\footnote{En nuestro caso el parámetro se fijo en 1/K}
\end{itemize}

\begin{enumerate}
\def\labelenumi{\arabic{enumi}.}
\setcounter{enumi}{1}
\item
Para cada país $d \in \{1,2,... D\}$
\end{enumerate}

\begin{itemize}
\item
generar un vector de proporciones de componentes
$\theta_d \sim Dir_K(\alpha)$ con $\alpha \in \mathcal{R}_{>0}^K$
un parámetro fijo\footnote{En nuestro caso el parámetro se fijo en 1/K}
\item
Para cada dólar exportado:

\begin{enumerate}
	\def\labelenumi{\roman{enumi}.}
	\item
	generar una asignación del componente $z_{dn} \sim Mult(\theta_d)$
	\item
	asignar el producto $w_{dn} \sim Mult(\beta_{zn})$
\end{enumerate}
\end{itemize}

En la figura \ref{fig:grafo_blei} se puede observar el proceso generativo de forma gráfica. Aquí, cada nodo representa una distribución de probabilidad. Las aristas significan que la distribución de salida definen los parámetros de la distribución de entrada. Los recuadros significan replicación: El recuadro interior representa que el proceso se realiza para cada dólar exportado en el país. El recuadro exterior representa que el proceso se realiza para cada
país en el corpus.


\begin{figure}
	\centering	
%	\includegraphics[width=.65\linewidth]{grafo}


\tikzset{every picture/.style={line width=0.75pt}} %set default line width to 0.75pt        

\begin{tikzpicture}[x=0.75pt,y=0.75pt,yscale=-1,xscale=1]
%uncomment if require: \path (0,300); %set diagram left start at 0, and has height of 300


%Shape: Circle [id:dp8853764007735601] 
\draw   (100,146) .. controls (100,132.19) and (111.19,121) .. (125,121) .. controls (138.81,121) and (150,132.19) .. (150,146) .. controls (150,159.81) and (138.81,171) .. (125,171) .. controls (111.19,171) and (100,159.81) .. (100,146) -- cycle ;
%Shape: Circle [id:dp33419535302461556] 
\draw   (205,146) .. controls (205,132.19) and (216.19,121) .. (230,121) .. controls (243.81,121) and (255,132.19) .. (255,146) .. controls (255,159.81) and (243.81,171) .. (230,171) .. controls (216.19,171) and (205,159.81) .. (205,146) -- cycle ;
%Shape: Circle [id:dp6232220468757862] 
\draw   (281,47) .. controls (281,33.19) and (292.19,22) .. (306,22) .. controls (319.81,22) and (331,33.19) .. (331,47) .. controls (331,60.81) and (319.81,72) .. (306,72) .. controls (292.19,72) and (281,60.81) .. (281,47) -- cycle ;
%Shape: Circle [id:dp49572645552205263] 
\draw  [fill={rgb, 255:red, 199; green, 199; blue, 199 }  ,fill opacity=1 ] (410,145) .. controls (410,131.19) and (421.19,120) .. (435,120) .. controls (448.81,120) and (460,131.19) .. (460,145) .. controls (460,158.81) and (448.81,170) .. (435,170) .. controls (421.19,170) and (410,158.81) .. (410,145) -- cycle ;
%Shape: Circle [id:dp04424520000997234] 
\draw   (311,145) .. controls (311,131.19) and (322.19,120) .. (336,120) .. controls (349.81,120) and (361,131.19) .. (361,145) .. controls (361,158.81) and (349.81,170) .. (336,170) .. controls (322.19,170) and (311,158.81) .. (311,145) -- cycle ;

%Shape: Rectangle [id:dp8518034256278205] 
\draw   (299.13,100.76) -- (496.19,100.76) -- (496.19,199.76) -- (299.13,199.76) -- cycle ;
%Shape: Rectangle [id:dp2099791938650164] 
\draw   (173,86) -- (515.13,86) -- (515.13,221.72) -- (173,221.72) -- cycle ;
%Straight Lines [id:da11406541597292297] 
\draw    (150,146) -- (203,146) ;
\draw [shift={(205,146)}, rotate = 180] [fill={rgb, 255:red, 0; green, 0; blue, 0 }  ][line width=0.75]  [draw opacity=0] (8.93,-4.29) -- (0,0) -- (8.93,4.29) -- cycle    ;

%Straight Lines [id:da3621589464216769] 
\draw    (255,146) -- (309,145.04) ;
\draw [shift={(311,145)}, rotate = 538.98] [fill={rgb, 255:red, 0; green, 0; blue, 0 }  ][line width=0.75]  [draw opacity=0] (8.93,-4.29) -- (0,0) -- (8.93,4.29) -- cycle    ;

%Straight Lines [id:da09879567766811748] 
\draw    (361,145) -- (408,145) ;
\draw [shift={(410,145)}, rotate = 180] [fill={rgb, 255:red, 0; green, 0; blue, 0 }  ][line width=0.75]  [draw opacity=0] (8.93,-4.29) -- (0,0) -- (8.93,4.29) -- cycle    ;

%Straight Lines [id:da23517788835593556] 
\draw    (328.92,60.5) -- (413.33,125.45) ;
\draw [shift={(414.92,126.67)}, rotate = 217.57] [fill={rgb, 255:red, 0; green, 0; blue, 0 }  ][line width=0.75]  [draw opacity=0] (8.93,-4.29) -- (0,0) -- (8.93,4.29) -- cycle    ;

%agrego a mano
\draw (211,47) circle (0.65cm);
\draw (237,47) -- (278,47);
\draw [shift={(281,47)}, rotate =180] [fill={rgb, 255:red, 0; green, 0; blue, 0 }  ][line width=0.75]  [draw opacity=0] (8.93,-4.29) -- (0,0) -- (8.93,4.29) -- cycle    ;


% Text Node
\draw (125,178) node  [align=left] {$\displaystyle \alpha $};
% Text Node
\draw (227.58,179) node  [align=left] {$\displaystyle \theta $};
% Text Node
\draw (338.42,177) node  [align=left] {$\displaystyle z$};
% Text Node
\draw (437.42,177) node  [align=left] {$\displaystyle w$};
% Text Node
\draw (305,12) node  [align=left] {$\displaystyle \beta $};
%agrego a mano
\draw (210,12) node  [align=left] {$\displaystyle \eta $};
% Text Node
\draw (484,189) node  [align=left] {$\displaystyle N$};
% Text Node
\draw (504,208) node  [align=left] {$\displaystyle M$};


\end{tikzpicture}

	\caption{fuente: \cite{blei2003latent}}
	\label{fig:grafo_blei}
\end{figure}



Un Proceso de Dirichlet es una familia de procesos estocásticos donde las realizaciones son ellas mismas distribuciones de probabilidad. Es decir, el rango de esta distribución (así como en una normal son los reales) son distribuciones de probabilidad. Para interpretarlo geométricamente, la figura \ref{fig:dirichlet} muestra un ejemplo de la distribución de densidad para 3 productos y 4 componentes. El triángulo representa todas las distribuciones (multinomiales)
posibles sobre los tres productos. Cada uno de los vértices del triángulo es una distribución de probabilidad que asigna una probabilidad de 1 a uno de los productos. El punto medio de cada lado, es una distribución con probabilidad 0.5 a dos componentes. El cuarto componente, el centroide del triángulo, asigna probabilidad de $\frac{1}{3}$ a cada producto. Los cuatro puntos marcado con x son las distribuciones multinomiales de $p(w|z)$  para cada uno de los cuatro componentes. La altura en el eje z es una posible distribución de densidad sobre el
simplex, es decir, sobre las distribuciones de densidad multinomiales, dada por LDA. el parámetro que define a la distribución de dirichlet (en nuestro caso, los parámetros $\eta$ y $\alpha$) determinan el grado de concentración de las distribuciones resultantes. Para una distribución $Dir(\alpha)$, $\alpha$ define el grado de simetría de las distribuciones multinomiales que el proceso genere. Con valores mucho menores a 1, las distribuciones resultantes estarán muy concentradas sobre algunos elementos, mientras que valores mucho mayores que 1 generarían distribuciones muy uniformes. En términos de nuestro problema, un $\alpha$ muy pequeño generará que en los países haya uno o unos pocos componentes característicos, mientras que un $\eta$ muy pequeño generará que la distribución que caracteriza a un componente sea muy asimétrica sobre los productos, y por lo tanto que haya unos pocos productos muy importantes, y el resto con probabilidad nula o casi nula.

\begin{figure}
	\centering	
	\includegraphics[width=.65\linewidth]{dirichlet}
	\caption{fuente: \cite{blei2003latent}}
	\label{fig:dirichlet}
\end{figure}

Cuando observamos los datos, no contamos con los tópicos ni con su
distribución, sino con los productos y países. El objetivo es realizar
inferencia sobre las variables latentes, mediante el Teorema de Bayes:

$$
p(\theta,z|w,\alpha,\beta) = \frac{p(\theta,z,w|\alpha,\beta)}{p(w|\alpha,\beta)}
$$

Nuestra función objetivo es:

$$
\ell(\alpha, \beta) = \sum_{d=1}^M \log p(w_d|\alpha,\beta)
$$

El problema es que esta ecuación es intratable, por la interacción entre $\theta$ y $\beta$. Por ello, la inferencia se realiza sobre una familia de modelos que se sabe que son una cota inferior de probabilidad, y que son tratables. Estos modelos tienen parámetros variacionales, que se ajustan para obtener el modelo que más se acerca a la cota inferior. La forma de obtener una familia de modelos tratables es considerar algunas modificaciones sobre el modelo gráfico original, removiendo nodos y aristas \citep{hoffman2013stochastic}. En la figura \ref{fig:grafo_variational} se puede observar la solución propuesta por \citep{blei2003latent} que, dado que utilizamos la implementación del modelo que se encuentra en \cite{scikit-learn} es también la del presente trabajo.

\begin{figure}[h]
	\centering	
%	\includegraphics[width=.65\linewidth]{grafo_variational}



\tikzset{every picture/.style={line width=0.75pt}} %set default line width to 0.75pt        

\begin{tikzpicture}[x=0.4pt,y=0.4pt,yscale=-1,xscale=1]
%uncomment if require: \path (0,300); %set diagram left start at 0, and has height of 300

%Shape: Circle [id:dp8853764007735601] 
\draw   (14,151) .. controls (14,137.19) and (25.19,126) .. (39,126) .. controls (52.81,126) and (64,137.19) .. (64,151) .. controls (64,164.81) and (52.81,176) .. (39,176) .. controls (25.19,176) and (14,164.81) .. (14,151) -- cycle ;
%Shape: Circle [id:dp33419535302461556] 
\draw   (119,151) .. controls (119,137.19) and (130.19,126) .. (144,126) .. controls (157.81,126) and (169,137.19) .. (169,151) .. controls (169,164.81) and (157.81,176) .. (144,176) .. controls (130.19,176) and (119,164.81) .. (119,151) -- cycle ;
%Shape: Circle [id:dp6232220468757862] 
\draw   (195,52) .. controls (195,38.19) and (206.19,27) .. (220,27) .. controls (233.81,27) and (245,38.19) .. (245,52) .. controls (245,65.81) and (233.81,77) .. (220,77) .. controls (206.19,77) and (195,65.81) .. (195,52) -- cycle ;
%Shape: Circle [id:dp49572645552205263] 
\draw  [fill={rgb, 255:red, 199; green, 199; blue, 199 }  ,fill opacity=1 ] (324,150) .. controls (324,136.19) and (335.19,125) .. (349,125) .. controls (362.81,125) and (374,136.19) .. (374,150) .. controls (374,163.81) and (362.81,175) .. (349,175) .. controls (335.19,175) and (324,163.81) .. (324,150) -- cycle ;
%Shape: Circle [id:dp04424520000997234] 
\draw   (225,150) .. controls (225,136.19) and (236.19,125) .. (250,125) .. controls (263.81,125) and (275,136.19) .. (275,150) .. controls (275,163.81) and (263.81,175) .. (250,175) .. controls (236.19,175) and (225,163.81) .. (225,150) -- cycle ;
%Shape: Rectangle [id:dp8518034256278205] 
\draw   (213.13,105.76) -- (411.6,105.76) -- (411.6,208) -- (213.13,208) -- cycle ;
%Shape: Rectangle [id:dp2099791938650164] 
\draw   (87,91) -- (440.6,91) -- (440.6,233) -- (87,233) -- cycle ;
%Straight Lines [id:da11406541597292297] 
\draw    (64,151) -- (117,151) ;
\draw [shift={(119,151)}, rotate = 180] [fill={rgb, 255:red, 0; green, 0; blue, 0 }  ][line width=0.75]  [draw opacity=0] (8.93,-4.29) -- (0,0) -- (8.93,4.29) -- cycle    ;

%Straight Lines [id:da3621589464216769] 
\draw    (169,151) -- (223,150.04) ;
\draw [shift={(225,150)}, rotate = 538.98] [fill={rgb, 255:red, 0; green, 0; blue, 0 }  ][line width=0.75]  [draw opacity=0] (8.93,-4.29) -- (0,0) -- (8.93,4.29) -- cycle    ;

%Straight Lines [id:da09879567766811748] 
\draw    (275,150) -- (322,150) ;
\draw [shift={(324,150)}, rotate = 180] [fill={rgb, 255:red, 0; green, 0; blue, 0 }  ][line width=0.75]  [draw opacity=0] (8.93,-4.29) -- (0,0) -- (8.93,4.29) -- cycle    ;

%Straight Lines [id:da23517788835593556] 
\draw    (242.92,65.5) -- (327.33,130.45) ;
\draw [shift={(328.92,131.67)}, rotate = 217.57] [fill={rgb, 255:red, 0; green, 0; blue, 0 }  ][line width=0.75]  [draw opacity=0] (8.93,-4.29) -- (0,0) -- (8.93,4.29) -- cycle    ;

%Shape: Rectangle [id:dp7501348302770449] 
\draw   (701,53.33) -- (803,53.33) -- (803,222.33) -- (701,222.33) -- cycle ;
%Shape: Rectangle [id:dp832082779713454] 
\draw   (551,46.33) -- (841,46.33) -- (841,237.33) -- (551,237.33) -- cycle ;
%Shape: Circle [id:dp8534167348661814] 
\draw   (610.39,157.5) .. controls (624.2,157.56) and (635.34,168.8) .. (635.28,182.61) .. controls (635.22,196.42) and (623.98,207.56) .. (610.17,207.5) .. controls (596.37,207.44) and (585.22,196.2) .. (585.28,182.39) .. controls (585.34,168.58) and (596.59,157.44) .. (610.39,157.5) -- cycle ;
%Shape: Circle [id:dp4466934905564389] 
\draw   (610.83,58.5) .. controls (624.63,58.56) and (635.78,69.8) .. (635.72,83.61) .. controls (635.66,97.42) and (624.41,108.56) .. (610.61,108.5) .. controls (596.8,108.44) and (585.66,97.2) .. (585.72,83.39) .. controls (585.78,69.58) and (597.02,58.44) .. (610.83,58.5) -- cycle ;
%Straight Lines [id:da2762808842403305] 
\draw    (610.61,108.5) -- (610.4,155.5) ;
\draw [shift={(610.39,157.5)}, rotate = 270.25] [fill={rgb, 255:red, 0; green, 0; blue, 0 }  ][line width=0.75]  [draw opacity=0] (8.93,-4.29) -- (0,0) -- (8.93,4.29) -- cycle    ;

%Shape: Circle [id:dp49555163496653243] 
\draw   (753.39,158.5) .. controls (767.2,158.56) and (778.34,169.8) .. (778.28,183.61) .. controls (778.22,197.42) and (766.98,208.56) .. (753.17,208.5) .. controls (739.37,208.44) and (728.22,197.2) .. (728.28,183.39) .. controls (728.34,169.58) and (739.59,158.44) .. (753.39,158.5) -- cycle ;
%Shape: Circle [id:dp6912113806172252] 
\draw   (753.83,59.5) .. controls (767.63,59.56) and (778.78,70.8) .. (778.72,84.61) .. controls (778.66,98.42) and (767.41,109.56) .. (753.61,109.5) .. controls (739.8,109.44) and (728.66,98.2) .. (728.72,84.39) .. controls (728.78,70.58) and (740.02,59.44) .. (753.83,59.5) -- cycle ;
%Straight Lines [id:da3696244463925229] 
\draw    (753.61,109.5) -- (753.4,156.5) ;
\draw [shift={(753.39,158.5)}, rotate = 270.25] [fill={rgb, 255:red, 0; green, 0; blue, 0 }  ][line width=0.75]  [draw opacity=0] (8.93,-4.29) -- (0,0) -- (8.93,4.29) -- cycle    ;


% Text Node
\draw (39,189) node  [align=left] {$\displaystyle \alpha $};
% Text Node
\draw (142.58,190) node  [align=left] {$\displaystyle \theta $};
% Text Node
\draw (252.42,188) node  [align=left] {$\displaystyle z$};
% Text Node
\draw (351.42,188) node  [align=left] {$\displaystyle w$};
% Text Node
\draw (184,49) node  [align=left] {$\displaystyle \beta $};
% Text Node
\draw (398,194) node  [align=left] {$\displaystyle N$};
% Text Node
\draw (423,217) node  [align=left] {$\displaystyle M$};
% Text Node
\draw (788,209) node  [align=left] {$\displaystyle N$};
% Text Node
\draw (825,222) node  [align=left] {$\displaystyle M$};
% Text Node
\draw (570.58,183) node  [align=left] {$\displaystyle \theta $};
% Text Node
\draw (569.58,81) node  [align=left] {$\displaystyle \gamma $};
% Text Node
\draw (713.58,82) node  [align=left] {$\displaystyle \varphi $};
% Text Node
\draw (714.42,181) node  [align=left] {$\displaystyle z$};


\end{tikzpicture}

	\caption{fuente: \cite{blei2003latent}}
	\label{fig:grafo_variational}
\end{figure}


La estimación de los parámetros se realiza a través del proceso de \emph{variational Expectation Maximization} (EM):

\begin{itemize}
\item \textbf{paso E}: Optimizamos los parámetros variacionales $\gamma, \varphi$
\item \textbf{paso M}: Para los valores fijos $\gamma, \varphi$, maximizamos la cota inferior respecto a los parámetros del modelo, $\alpha,\beta$
\end{itemize}

Estos dos pasos se alternan hasta converger. Por último, se realiza un suavizado sobre las probabilidades que un componente asigna a un producto, para que sean siempre mayores a 0.



\paragraph{reflexiones finales}
Por último, es interesante detenerse en la factibilidad del modelo dado el cambio de dominio del problema. La naturaleza tan distinta de los datos usados tradicionalmente en minería de textos y Topic Modelling, respecto de los de comercio internacional, amerita el interrogante respecto de la posibilidad de que el modelo pueda operar en el nuevo dominio. Sin embargo, en términos de la estructura de datos, ambos problemas tienen más semejanzas de las que aparentan. En primer lugar, La dimensión tradicional del problema es $NxV$ ($N$ observaciones, en el orden de magnitud de miles, $V$ el vocabulario, también en el orden de magnitud de los miles). En este caso, el problema es aproximadamente de dimensión $NxP$, donde las $N$ observaciones son los pares año-país, con 180 países y 54 años, y $P$ productos, que en SITC a 4 dígitos son unos 750 aproximadamente. Es decir, estamos en un orden de magnitud similar al de un dataset pequeño en un problema tradicional de Topic Modelling. 
Respecto de la distribución, en el vocabulario suele operar la ley de Zipf \citep{Powers1998},es decir una distribución fuertemente asimétrica de la frecuencia de las palabras. En nuestro problema también existe una distribución asimétrica de los productos exportados por país, aunque no necesariamente del mismo orden de magnitud. Por último, un cambio importante en ambos dominios es la diferencia entre la frecuencia de las palabras en un texto (decenas o centenas, según el tamaño) y los valore exportados de los países (miles de millones). Esta diferencia en principio no debería afectar al modelo, dado que lo que el modelo considera en su optimización son las distribuciones entre los distintos elementos (frecuencias de palabras o valores exportados por producto) y no los valores absolutos.


\subsection{Grafo Bipartito}

Otra manera de analizar el comercio internacional considerando la dimensión \textit{producto} es mediante técnicas de análisis de redes. Es difundido en la literatura el uso de grafos bipartitos para este análisis \citep{10.1371/journal.pone.0197575,straka2017grand,ferreira2016topology,caldarelli2012network}. En la figura \ref{fig:bipgraph} se puede observar un diagrama de un grafo bipartito de países, \textit{a}, y productos, \textit{b}. Una arista entre un país $a_c$ y un producto $b_i$ representa en este caso que el país $a_c$ exporta el producto $b_i$. Dado que tanto los países como los productos se representan como nodos, la estructura de grafo bipartito nos permite mantener la diferencia cualitativa entre ambos tipos de vértices.


\begin{figure}
	\centering
\begin{tikzpicture}

\begin{scope}[rotate=90]
\SetVertexMath
\grEmptyLadder[RA=1,RB=2]{4}   
\end{scope}
\Edges(b2,a0,b0,a1,b2,a3,b1,a2)
\end{tikzpicture} 

\caption{Grafo bipartito, países y productos} \label{fig:bipgraph}
\end{figure}


Tal como sucedía con el comercio agregado a nivel países, los tamaños relativos de las economías nacionales implican que un mismo monto exportado tenga un significado muy distinto según el país del cual provienen. A su vez, como los volúmenes comerciados de los diferentes productos también varían fuertemente, lo anterior aplica a este otro tipo de nodos. Es por ello que en la literatura se utiliza una normalización de las exportaciones, conocida como \textit{Ventajas Comparativas Reveladas}, o \textit{RCA} por sus siglas en inglés. La misma fue propuesta por \cite{balassa1965trade} y tiene la siguiente forma funcional:

$$
\operatorname{RCA}(c, i)= \frac{\displaystyle \frac{x(c, i)}{\sum_{i} x(c, i)}}{\displaystyle \frac{\sum_{c} x(c, i)}{\sum_{c, i} x(c, i)}}
$$

Dónde $x(c,i)$ es el valor de las exportaciones del \textit{país c} en \textit{producto i}. El numerador indica la proporción que dicho producto representa en las exportaciones totales del país c. El denominador indica la proporción que este mismo producto representa en las exportaciones totales de todos los países. Es decir, \textit{RCA} muestra la relación en la importancia de un producto $i$ en un país $c$ respecto de la importancia promedio de ese producto en la economía mundial. De esta forma, un RCA alto implica que el país en cuestión tiene \textit{ventaja relativa} para exportar un producto, mientras que un RCA bajo implica que el país tiene una \textit{desventaja relativa} con el producto. De esta forma, se puede construir un grafo no ponderado estableciendo el punto de corte en $RCA>1$. 

Habiendo establecido nodos y aristas del grafo bipartito, el problema reside en que la mayor parte de las técnicas de análisis sobre grafos están definidas para grafos simples, no para grafos bipartitos. Por ello se continúa por realizar la proyección sobre uno de los tipos de nodos para luego analizar los resultados \citep{zhou2007bipartite}. La proyección se realiza de forma ponderada y el peso asignado a una arista entre dos nodos es la cantidad de nodos del otro tipo con los cuales estaban conectados ambos nodos en el grafo original. Esto significa, en la proyección del grafo de países, que aquellos que compartían ventajas comparativas reveladas sobre un conjunto grande productos estarán fuertemente conectados, mientras que los países que poseen canastas exportadoras muy diferentes, estarán débilmente conectados.

Este procedimiento se realizó para los nodos-país. El resultados es un grafo de características similares al elaborado en el capítulo anterior, pero que en esta oportunidad los vínculos entre países no reflejan sus vínculos comerciales, sino su similitud respecto a la estructura productiva exportadora. A diferencia del grafo de relaciones comerciales, aquí las métricas de centralidad no resultan particularmente interesantes porque el centro del grafo esta poblado por aquellos países que exportan las mercancías más comunes, mientras que en los márgenes se encuentran aquellos cuya canasta exportadora resulta muy diferente a la media. Lo que resulta particularmente interesante son las comunidades que se generan en el grafo, para ver si las mismas tienen alguna relación con lo dicho en la literatura respecto de la Nueva División Internacional del Trabajo \citep{frobel1978new}. Para ello se utilizaron técnicas de detección de comunidades. En particular, se utilizo el \textit{clustering de Louvain} \citep{blondel2008fast},donde se optimiza la modularidad de los clusters de forma \textit{greedy}. También se utilizó el método de detección de comunidades \textit{Walktrap} \citep{pons2005computing}, que se basa en la noción de que al realizar \textit{random walks} en el grafo, estos caminos tienden a quedarse atrapados en las secciones más densas del grafo, que corresponden a las comunidades. En ambas técnicas en número de comunidades encontradas es definido de forma automática por el algoritmo. 

Para el análisis de los productos, nos basamos en el concepto de \textit{proximidad} de \cite{Hidalgo2009,Hidalgo2009a,Hidalgo2007} definido como:

$$
\phi_{ij} = min (P(RCA_i>1/RCA_j>1),P(RCA_j>1/RCA_i>1))
$$

dónde $P(RCA_i/RCA_j)$ es la probabilidad condicional de exportar el producto $i$ dado que exporta el producto $j$. Es decir:

$$
P(RCA_i >1 /RCA_j >1) = \frac{P(RCA_i >1 \cap RCA_j >1)}{P(RCA_j >1)}
$$

$$
\text{con } P(RCA_j >1)= \frac{ \sum_{c} I(RCA_c >1)}{N}
$$

siendo N la cantidad de países y $\sum_{c}$ la sumatoria en los países. Es decir, la proporción de países en los cuales el producto $j$ tiene un $RCA>1$. Por su parte: 

$$
P(RCA_i >1 \cap RCA_j >1) = \frac{\sum_c I(RCA_i >1) \cap I(RCA_j >1)}{N}
$$

Esto es, la probabilidad de que un país tenga para ambos productos ventajas comparativas reveladas. Por lo tanto, $\phi_{ij}$ establece que dos productos son similares si son exportados por los mismos países. Dado que existen productos que son ubicuos, es decir que son exportados por la mayoría de los países, y son muchos quienes tienen un $RCA>1$, y que una métrica de distancia debe cumplir con el principio de simetría ($d(x,y)=d(y,x)$), esta función toma el mínimo de ambos valores. 


$\Phi$ es por lo tanto una matriz de distancias entre los productos. Esto resulta ideal para analizar los datos a través de la técnica de clustering \textit{K-medioids} \citep{kaufman1987clustering}. Este método construye K clusters de forma iterativa modificando el centro del cluster para optimizar la bondad de ajuste. A diferencia de K-means \citep{macqueen1967some} el centro de cada cluster se define en una observación existente en el dataset, y por lo tanto se puede recuperar a las mismas para caracterizar el cluster. En es caso del espacio de productos esto resulta de utilidad para comprender los grupos que se conforman. Finalmente, dado que se presentan los datos en un \textit{heatmap}, se decidió también utilizar la técnica de clustering jerárquico propuesta por \cite{ward1963hierarchical} para su ordenamiento.

\section{Resultados}

\subsection{Latent Dirichlet Allocation Models}

\subsubsection{Análisis de Componentes}


Los resultado obtenidos de los modelos de Topic Modelling para el análisis de textos normalmente se analizan en dos etapas: En primer lugar se etiquetan los tópicos obtenidos sobre la base de las palabras más salientes de cada tópico. Luego, se analiza su distribución en los textos. El etiquetado de los tópicos es una tarea subjetiva donde lo que se busca es un concepto generalizador de aquellas palabras que componen al tópico. Es posible, y frecuente, que esta tarea no se pueda realizar por falta de un concepto generalizador dentro del tópico. El hiperparámetro $k$, es decir la cantidad de tópicos, juega un rol fundamental en este punto, dado que con una cantidad baja de tópicos estos tenderán a reflejar conceptos amplios, mientras que si $k$ es mayor que la cardinalidad del espacio latente que se busca, esto puede generar tópicos repetidos o sobre-específicos. 
Lo anterior no escapa al presente dominio sino que se refuerza, dado que la búsqueda, subjetiva, de un concepto abarcador entre productos puede resultar más compleja que la de un concepto generalizador de un grupo de palabras. Un problema que no existe en el presente dominio es el de la polisemia, dado que todos los significantes, indices del nomenclador, hacen referencia a un único significado no ambiguo. Sin embargo, nuevos problemas aparecen respecto a este punto, como qué nomenclador de base utilizar y en qué nivel de desagregación. Por motivos de comparabilidad con los resultados de \cite{molinari2016especializacion} e \cite{Hidalgo2009} se decidió utilizar el nomenclador SITC a 4 dígitos \citep{un2006standard}. A su vez, se realizaron pruebas para varios valores de $k$: 2,4,6,8,10,20,30,40,50, 100 y 200, así como para diferentes valores de los hiperparámetros $\eta$ y $\alpha$ \footnote{dado que los resultados son robustos a estos últimos parámetros, se decidió utilizar $\alpha=\eta=\frac{1}{k}$ para las corridas con distintos valores de k}. Para el etiquetado de los componentes se observó que la práctica usual de observar los primeros diez elementos de la distribución no bastaba para encontrar una etiqueta generalizadora, por lo que se elaboró un tablero dinámico \footnote{ver \url{https://treemaps.shinyapps.io/LDA_worldtrade/}} para estudiar la distribución y su función acumulada, a la vez que se gráfico la distribución en función de un nomenclador de complejidad tecnológica \citep{lall2000technological}. En la tabla \ref{tabla:lall} se puede observar dichas categorías.  

\begin{table}
	\centering
	\begin{tabular}{ll}
		\hline
		 Código & Descripción \\ 
		\hline
		01 & Primary products \\ 
		02 & Resource-based manufactures: agro-based \\ 
		03 & Resource-based manufactures: other \\ 
		04 & Low technology manufactures: textile, garment and footwear \\ 
		05 & Low technology manufactures: other products \\ 
		06 & Medium technology manufactures: automotive \\ 
		07 & Medium technology manufactures: process \\ 
		08 & Medium technology manufactures: engineering \\ 
		09 & High technology manufactures: electronic and electrical \\ 
		10 & High technology manufactures: other \\ 
		99 & Unclassified products \\ 
		\hline
	\end{tabular}
\caption{Agrupamiento por complejidad tecnológica de \cite{lall2000technological}}
\label{tabla:lall}
\end{table}

A título ilustrativo, en la figura \ref{fig:top_dist} se presentan una captura de la interfaz elaborada para la caracterización de los componentes en el caso de $k=2$. Allí se presenta el código de SITC a 4 dígitos, la descripción del producto, su proporción y la proporción acumulada, ordenados de forma decreciente según su peso. En la figura \ref{fig:top_dist1} se puede observar que la distribución del primer componente asigna un gran peso al petróleo crudo, y que los siguientes productos, además de aquellos no especificados, son también derivados del petróleo, como el gasoil, gas propano (petroleum gases), etc. Una etiqueta plausible para este componente sería \textit{Petróleo y derivados}. Nótese que también se encuentras otros productos como el carbón (coal) y metales como el hierro (iron) oro (gold) y cobre (copper).
En la figura \ref{fig:top_dist2} se presenta la distribución del segundo componente. Allí no se observa una distribución tan asimétrica como en el caso del primer componente, sino que el primer producto pesa tan solo un 5\%. Los productos más destacados son los vehículos de pasajeros, microcircuitos electrónicos, partes y accesorios, etc. Podría decirse que este componente representa a los productos de origen industrial en general.


\begin{figure}
	\centering
	\subfigure[Componente 1]{\label{fig:top_dist1}\includegraphics[width=\linewidth]{comp1k2}}
	\subfigure[Componente 2]{\label{fig:top_dist2}\includegraphics[width=\linewidth]{comp2k2}}
	\caption{Distribución de los componentes según Lall. K=2}
	\label{fig:top_dist}
\end{figure}

La figura \ref{fig:lall} muestra la distribución de los componente siguiendo la clasificación de \cite{lall2000technological}. Allí se observa cómo el primer componente esta constituído esencialmente por productos primarios y manufacturas derivadas de los mismos. El segundo componente presenta una distribución más uniforme, donde se destacan las manufacturas de tecnologías media y alta (ingenieriles y electrónicas)

\begin{figure}
	\centering
	\subfigure[Componente 1]{\label{fig:lall_1}\includegraphics[width=0.45\linewidth]{graficoLall_k2_comp1}}
	\subfigure[Componente 2]{\label{fig:lall_1}\includegraphics[width=0.45\linewidth]{graficoLall_k2_comp2}}
	\caption{Distribución de los componentes según Lall. K=2}
	\label{fig:lall}
\end{figure}


Vale notar que los productos de origen agropecuario, ganadero, forestal, etc. no quedan bien definidos en $k=2$. A su vez, resulta interesante como, bajo la consigna de dividir es espacio de productos en dos grupos, el modelo de LDA encuentra su óptimo en esta diferenciación. Por un lado, reproduce la dicotomía clásica de países industriales versus países productores de materias primas \citep{coffey1996newer,hutchinson2004globalisation}. Por otro lado, dicha imagen tradicional pone a los productos agropecuarios como unos de los polos, y no exclusivamente al petróleo y sus derivados. Esto puede entenderse a partir del rol particular que la producción petrolera juega en la estructuración de las economías nacionales, y en particular de las exportaciones \citep{ross2012oil,carrera2017renta,starosta2016new}. En ese sentido, las formas tan particulares de las exportaciones de los países petroleros llevan a que el modelo encuentre su óptimo construyendo uno de los dos componentes con estos productos. 


una de las problemáticas observadas en el análisis de los diferentes ejercicios es definir la granularidad correcta del objeto de estudio. Como se mencionó arriba, el parámetro $k$ define cuan específicos o genéricos son los componentes obtenidos. El problema surge en que en el análisis económico aquellos fenómenos que resulta de interés explorar pueden encontrarse a diferentes niveles de granularidad. Esto es, desde el punto de vista económico puede resultar de interés seguir el comportamiento de un componente que refiera a \textit{derivados de la soja}, pero para poder dar con el mismo, sería necesario un nivel de granularidad que puede implicar que otro concepto interesante como \textit{productos textiles} se multiplique a lo largo de varios componentes. 

La figura \ref{fig:componentes} muestra la distribución de los componentes \footnote{sin ponderar por el peso de las exportaciones, es decir de forma equiponderada entre países}. Allí se observa que la distirbución es irregular, y que tiende a haber componentes que se destacan por sobre los demás, como el primer componente para k igual a 2,4 y 6 ; el componente octavo en k igual a 8,10,20, etc. Estos componentes están compuestos o bien por productos petroleros o por una mezcla de estos con otros productos primarios. Esto se puede interpreta como que, o bien son más los países productores primarios, o bien estos productos tienen un mayor peso en las canastas exportadoras de aquellos países que los producen, o una combinación de ambos fenómenos.

\begin{figure}
	\centering	
	\includegraphics[width=0.65\linewidth]{componentes}
	\caption{Distribución promedio de los componentes en los países, varios valores de k}
	\label{fig:componentes}
\end{figure}


Finalmente, para elaborar las etiquetas finales se consultó especialistas sectoriales. En la tabla \ref{tabla:desc_comp} se pueden ver las etiquetas para el modelo con $k=30$, que se utilizará para mostrar los resultados, dado que permite una buena granularidad para los objetivos del presente trabajo. Allí se presenta una descripción general por rama, excepto para el componente 19 donde esto no es posible, además se presenta como \textit{subgrupo} algún conjunto característico de productos que permite una mejor especificación, y para el caso de los productos industriales, el nivel de complejidad, siguiendo la clasificación de \cite{lall2000technological}. Por último, en la columna \textit{país} se muestra aquel país para el que tiene una mayor participación del componente en el promedio de la serie. Un caso particular es el componente 5, que posee productos de alta complejidad para el comienzo de la serie, pero que luego cayeron en desuso o disminuyeron su participación en el comercio internacional. Por ejemplo, las cintas de grabación, líneas de teléfono o el papel fotográfico. Es natural que el país que caracteriza a dicho componente sea aquel cuya serie es de menor duración, como Checoslovaquia, que por disolverse en el año 92', su denominador es menor que en los demás países. 

\begin{table}
	\centering
	\begin{tabular}{rllll} 
		\hline
		\multicolumn{1}{l}{\textbf{Comp}} & \textbf{Grupo}    & \textbf{Subgrupo}                                                                                                                & \textbf{Complejidad} & \textbf{País}                                                      \\ 
		\hline
		1                                 & Minerales         & Carbón, hierro                                                                                                                   & -                    & Australia                                                          \\ 
		\hline
		2                                 & Industria         & Textiles, ingenieril, otros                                                                                                      & Baja y media         & San Marino                                                         \\ 
		\hline
		3                                 & Industria         & Automoviles                                                                                                                      & Media                & Bélgica                                                            \\ 
		\hline
		4                                 & Industria         & Textil, juguetes, etc                                                                                                            & Baja                 & Macao                                                              \\ 
		\hline
		5                                 & Industria         & \begin{tabular}[c]{@{}l@{}}Electrónicos no digitales.\\Cintas de grabación.\\Lineas de telefono.\\Papél fotográfico\end{tabular} & Alta (hasta 70')     & Checoslovaquia                                                     \\ 
		\hline
		6                                 & Industria         & \begin{tabular}[c]{@{}l@{}}Autos, barcos, \\maquinaria\end{tabular}                                                            & Media y alta         & Japón                                                              \\ 
		\hline
		7                                 & Petróleo          & Estado Gaseoso                                                                                                                   & -                    & Turkmenistán                                                       \\ 
		\hline
		8                                 & Minerales         & Cobre                                                                                                                            & -                    & Chile                                                              \\ 
		\hline
		9                                 & Agropecuario      & Café, bananas, otros                                                                                                             & -                    & Réunion                                                            \\ 
		\hline
		10                                & Industria         & Autos y electrónicos                                                                                                             & Media y alta         & México                                                             \\ 
		\hline
		11                                & Industria         & \begin{tabular}[c]{@{}l@{}}Autos, partes y\\ componentes\end{tabular}                                                            & Media y alta         & Alemania                                                           \\ 
		\hline
		12                                & Petróleo          & Crudo                                                                                                                            & -                    & Sudán del Sur                                                      \\ 
		\hline
		13                                & -                 & Oro, relojes, joyas                                                                                                              & -                    & Suiza                                                              \\ 
		\hline
		14                                & Industria         & Químicos                                                                                                                         & Alta                 & Curazao                                                            \\ 
		\hline
		15                                & Minerales         & Diamantes                                                                                                                        & -                    & Botswana                                                           \\ 
		\hline
		16                                & Industria + Agro. & \begin{tabular}[c]{@{}l@{}}Aviones, autopartes,\\ soja y maíz\end{tabular}                                                       & Media y alta         & USA                                                                \\ 
		\hline
		17                                & Industria + Agro. & \begin{tabular}[c]{@{}l@{}}Autopartes,\\ madera y derivados\end{tabular}                                                         & Media                & Finlandia                                                          \\ 
		\hline
		18                                & Industria + Agro. & \begin{tabular}[c]{@{}l@{}}Productos primarios\\ y textiles\end{tabular}                                                         & Baja                 & Isla de Navidad                                                    \\ 
		\hline
		19                                & -                 & \begin{tabular}[c]{@{}l@{}}Transacciones especiales,\\ no clasificadas\end{tabular}                                              & -                    & Isla San Martín                                                    \\ 
		\hline
		20                                & Combustibles      & Fuel oil, gasolina, etc.                                                                                                         & -                    & \begin{tabular}[c]{@{}l@{}}Rep. Dem.\\ Pop.del Yemen\end{tabular}  \\ 
		\hline
		21                                & Industria         & \begin{tabular}[c]{@{}l@{}}Medicamentos e\\ instrumentos médicos.\end{tabular}                                                   & Alta                 & Irlanda                                                            \\ 
		\hline
		22                                & Petróleo + Agro   & \begin{tabular}[c]{@{}l@{}}Hidrocarburos,\\aceite de palma,cacao, etc.\end{tabular}                                              & -                    & Ghana                                                              \\ 
		\hline
		23                                & Industria         & \begin{tabular}[c]{@{}l@{}}Procesadores,\\microcircuitos,\\ juguetes y calzado.\end{tabular}                                     & Alta y baja          & China                                                              \\ 
		\hline
		24                                & Agropecuario      & \begin{tabular}[c]{@{}l@{}}Carnes, pescados,\\ lácteos\end{tabular}                                                              & -                    & Islandia                                                           \\ 
		\hline
		25                                & Minerales + Agro. & Soja y Hierro                                                                                                                    & -                    & Paraguay                                                           \\ 
		\hline
		26                                & Industria + Agro. & \begin{tabular}[c]{@{}l@{}}Aviones, perfumería,\\ vino, queso\end{tabular}                                                       & Alta                 & Francia                                                            \\ 
		\hline
		27                                & Industria         & Microcircuitos                                                                                                                   & Alta                 & filipinas                                                          \\ 
		\hline
		28                                & Industria + Agro. & \begin{tabular}[c]{@{}l@{}}Arroz, algodón,\\textiles, goma, etc.\end{tabular}                                                 & Baja                 & Pakistán                                                           \\ 
		\hline
		29                                & Industria + Agro. & maquinaria, flores, quesos.                                                                                                      & Alta                 & Países Bajos                                                       \\ 
		\hline
		30                                & Industria         & Autos, motores, etc.                                                                                                             & Media y alta         & Reino Unido                                                        \\
		\hline
	\end{tabular}
	\caption{Descrición Componentes. K = 30}
	 \label{tabla:desc_comp}
\end{table}


\subsubsection{Análisis de países}


Con los componentes definidos, es posible analizar la composición de la canasta exportadora de cada país. Dado que en la definición del problema nuestra unidad es el par país\&año, eso significa que podemos comparar la evolución en la distribución de los componentes al interior de cada país. 


\paragraph{Países petroleros}


En la figura \ref{fig:graficoldak30irnirqkwtsauven} se observa la distribución de los componentes para los cinco países fundadores de la Organización de Países Exportadores de Petróleo (OPEP) en 1960. Tal como se refleja en el gráfico, las exportaciones de estos países se concentran en el petróleo y derivados. Es interesante notar cómo, previo a la crisis del petróleo de 1979 \citep{venn2016oil}, las exportaciones se dividían simétricamente entre los componentes 12 (petróleo crudo) y 20 (combustibles), sin embargo, luego de este episodio, con la suba de los precios del barril de crudo, la participación del componente 12 aumenta fuertemente, y se mantiene de esta forma hasta la actualidad. El caso de Venezuela es un tanto particular, dado que las exportaciones de combustible previo a la crisis del 79' tenían un mayor peso que el crudo, y si bien esto se revierte luego de la crisis, durante las décadas de los 80' y 90', el componente 20 sigue teniendo un peso importante, aunque el petróleo crudo va tendencialmente ganando participación hasta la actualidad. 

\begin{figure}
	\centering
	\includegraphics[width=0.7\linewidth]{../desagregado/LDA/graficos/graficoLDA_k30_IRN_IRQ_KWT_SAU_VEN}
	\caption{Distribución componentes: Irán, Irak, Kuwait, Arabia Saudita y Venezuela.}
	\label{fig:graficoldak30irnirqkwtsauven}
\end{figure}

\paragraph{América}

En la figura \ref{fig:graficoldak30canmexusa} se presenta la evolución de los componentes de Canadá, México y Estados Unidos. Los tres países tienen un comportamiento diferente. En Canada predomina el componente 17, una combinación de productos industriales de complejidad media y agropecuarios (madera). Con el comienzo del siglo XXI, este componente pareciera perder terreno frente al componente 12 (petróleo). México comienza la serie sin una especificidad clara, con algún peso mayor del componente 28 (Industrial de baja complejidad y agro) que rápidamente pierde terreno. Durante la década del 80', y parcialmente en los 90' cobra relevancia el componente 12 (petróleo crudo), pero al contrario de lo que sucede en Canadá, este componente pierde peso frente al componente 10 (industrial de media y alta complejidad) que a fines de los 80' se convierte en el componente con más peso, y continúa creciendo en importancia hasta la actualidad. Este proceso es analizado en la literatura como la \textit{Maquila Mexicana} \citep{sklair2012assembling}. En estos análisis se observa como las exportaciones de productos de alta complejidad por parte de México tiene como contrapartida la importación de los componentes desde Estados Unidos, siendo la \textit{maquilas} fábricas de baja complejidad, donde se hace uso de la mano de obra barata en México para ensamblar los componentes importados, y volver a enviarlos a Estados Unidos. Es interesante notar como la metodología propuesta en esta sección lo logra captar este fenómeno en su totalidad, generando una apariencia engañosa respecto de la estructura productiva mexicana. 

Estados Unidos comienza la serie con una predominancia de los componentes 5 (Electrónicos,analógicos) y 16 (Industria de alta complejidad y producción agropecuaria). Entre 1977 y 1978 cae la participación del componente 5 cae abruptamente, para luego seguir su derrotero hasta la actualidad. El componente 5 marca productos analógicos, considerados de alta complejidad hasta el quiebre producido por los cambios técnicos introducidos por las computadoras personales y las telecomunicaciones. Solo a modo de ejemplo, en junio de 1977 se crea la Apple II, las primeras computadoras personales de venta masiva. Es decir, hasta fines de los 70' la producción de Estados Unidos se dividía entre el agro y productos electrónicos no digitales. Sin embargo, en el componente 16, si bien aparece la industrial pesada (aviones, motores, maquinaria), no aparecen los microprocesadores, transistores y demás productos propios de la \textit{era digital}. Es decir que no existe simplemente un reemplazo tecnológico. Independientemente del punto de quiebre, este proceso puede asociarse a lo que en la literatura es ampliamente conocido como \textit{globalización}, donde a raíz de las posibilidades que brinda el mencionado cambio técnico, ciertos procesos productivos, como la producción de productos electrónicos, se trasladó hacia los países \textit{periféricos} \citep{henderson2002globalisation}.

\begin{figure}
	\centering
	\includegraphics[width=0.7\linewidth]{../desagregado/LDA/graficos/graficoLDA_k30_CAN_MEX_USA}
	\caption{Distribución componentes: Canadá, México y Estados Unidos.}
	\label{fig:graficoldak30canmexusa}
\end{figure}

En la figura  \ref{fig:graficoldak30bolchlper} se puede observar la evolución de los componentes para Bolivia, Chile y Perú. En el caso de Bolivia, este país comienza la serie con una fuerte predominancia del componente 8 (cobre). Sin embargo,desde mediados de los 60' la importancia de este componente empieza a disminuir, y a crecer la importancia del componente 7 (gas), primero como un pico en la década del 80', y luego de forma sostenida a partir del cambio de siglo. Estos resultados son coincidentes con lo conocido por la bibliografía especializada para dicho país \citep{auty2002sustaining, chavez2016can}. Por su parte, tanto en Chile como en Perú predomina la producción de Cobre y derivados, aunque en el caso peruano existe una variedad de otros componentes con alguna importancia menor, mientras que para Chile las exportaciones son casi en su totalidad sobre el componente \citep{moran2014multinational,mikesell2013world}.

\begin{figure}
	\centering
	\includegraphics[width=0.7\linewidth]{../desagregado/LDA/graficos/graficoLDA_k30_BOL_CHL_PER}
	\caption{Distribución componentes: Bolivia, Chile y Perú.}
	\label{fig:graficoldak30bolchlper}
\end{figure}

La figura \ref{fig:graficoldak30argbraury} muestra la evolución de los componentes de Argentina, Brasil y Uruguay. En el caso argentino se observa una predominancia del componente 25 (productos primarios como soja y hierro). Al comienzo de la serie también tiene relevancia el componente 24 (ganadería, pesca y lácteos), pero su peso disminuye en el tiempo. Viendo el detalle de los componentes de menor peso, se puede observar que a partir de los 90' crecen los componentes 3 (automóviles) y 14 (químicos), probablemente debido a la creación del Mercado Común del Sur, y las nuevas exportaciones regionales que ello trajo \citep{bekerman2010integracion}. En el caso Brasilero, se observa un cambio de componente dominante, desde el 9 (café, bananas) hacia el mencionado componente 25. Esto marca la misma tendencia que en el caso argentino a que predomine en este grupo de países la producción de commodities agrícolas. Por su parte, es posible que sea la influencia de Brasil la que determina el peso del hierro en el componente \citep{costantino2013gatopardismo}. A su vez, al comienzo de la serie destaca el componente 28 (industria de baja complejidad), que pierde importancia en el tiempo, a la vez que se observa un crecimiento a partir de los 2000' de los componentes 11 (autos, partes y componentes) y 12 (petróleo), aunque siempre de importancia secundaria respecto del componente 25. 
En el caso de Uruguay, a diferencia de Argentina y Brasil, el componente más importante es el 24 (ganadería, pesca y lácteos). Sin embargo, el mismo pierde importancia, a la vez que desde los 2000' comienzan a crecer las exportaciones en el componente 25, que desde 2009 es el de mayor peso \citep{redo2012}. 
En los tres países podemos observar una tendencia a la creciente exportación de commodities, en particular de la soja, y en el caso de Brasil del hierro.

\begin{figure}
	\centering
	\includegraphics[width=0.7\linewidth]{../desagregado/LDA/graficos/graficoLDA_k30_ARG_BRA_URY}
	\caption{Distribución componentes: Argentina, Brasil y Uruguay.}
	\label{fig:graficoldak30argbraury}
\end{figure}

\paragraph{Oceanía}

La figura \ref{fig:graficoldak30ausnzl} muestra la distribución de los componentes en Australia y Nueva Zelanda. En el caso australiano se marca una clara predominancia del componente 1 (Carbón y hierro), lo cual es esperable, dado que este país es el mayor exportador mundial de dichos minerales \citep{balat2009coal}. A su vez, la literatura especializada en dicho país menciona un \textit{boom} de los términos de intercambio para estos commodities, que permitiría explicar la tendencia ascendente a partir del año 2000 del componente 1 \citep{mckissack2008structural}. Nueva Zelanda, por su parte, refleja una predominancia del componente 24 (Agropecuario, carnes y lácteos), lo cual se corresponde con la especialización productiva que señala la literatura \citep{macleod2006intensification,ballingall2004farming}. Es interesante notar como, a diferencia de lo observado para el caso de Argentina, Brasil y Uruguay, para Austrialia y Nueva Zelanda no se observa un cambio en la especialización productiva. El caso australiano, dado que se trata de extracción de minerales, es lógico compararlo con el caso de Bolivia, Chile y Perú, en donde estos dos últimos tampoco muestran signos de cambio en su especialización para el mercado mundial. Sin embargo, Nueva Zelanda exporta sobre el componente 24, que es el mismo que definía el perfil exportador de Uruguay y jugaba un rol importante en Argentina para el principio de la serie. No obstante esto último, no se observa para Nueva Zelanda un incremento en la participación del componente 25 sobre sus exportaciones.

\begin{figure}
	\centering
	\includegraphics[width=0.7\linewidth]{../desagregado/LDA/graficos/graficoLDA_k30_AUS_NZL}
	\caption{Distribución componentes: Australia y Nueva Zelanda.}
	\label{fig:graficoldak30ausnzl}
\end{figure}

\paragraph{Europa}


En la figura \ref{fig:graficoldak30hunpolrou} se presentan los resultados para Hungría, Polonia y Rumanía, tres países que formaron parte de la zona de influencia de la Unión Soviética \citep{shama2000determinants}, y el \textit{Council for Mutual Economic Assistance} (CMEA). En el caso de Hungría, se observa que entre mediados de los 60' hasta mediados de los 70' tiene un peso importante el componente 5(industrial, electrónicos analógicos), que luego cae abruptamente. Este proceso se debe a que los productos del componente 5 se vuelven obsoletos dado el cambio técnico del período. Sin embargo, si se tratara simplemente de un cambio técnico, sería esperable que el componente 5 fuera reemplazado por otro que reflejase las nuevas tecnologías, lo cual no sucede. Esto puede estar relacionado con el deterioro de los términos de intercambio, tanto con el bloque socialista, como con los países de Europa Occidental, y la mayor apertura comercial con estos últimos \citep{koves1983hungarian}. Más allá de este punto, lo que se observa es una irregularidad que da cuenta de una falta de especialización productiva, en particular hasta mediados de los 90'. Con la caída de la Unión Soviética y la fuerte reestructuración económica dentro de estos países, se observa una creciente especialización sobre el componente 10 (industria, automotriz y electrónica)\citep{linden1998building}. Para el caso de Rumanía se observa además un pico del componente 4 (industria liviana, textiles y juguetes) que a partir de 2002 disminuye en importancia para dar lugar al componente 10. 


\begin{figure}
	\centering
	\includegraphics[width=0.7\linewidth]{../desagregado/LDA/graficos/graficoLDA_k30_HUN_POL_ROU}
	\caption{Distribución componentes: Hungría. Polonia y Rumanía.}
	\label{fig:graficoldak30hunpolrou}
\end{figure}

La figura \ref{fig:graficoldak30autczedeu} presenta la evolución de Austria y Alemania desde 1962, y República Checa desde su fundación como estado independiente a principios de los 90'. Tanto en el caso de Austria como Alemania, se observa una fuerte caía a fines de los años 70' del mencionado componente 5. Este fenómeno ya fue comentado en el caso de Estados Unidos y Hungría, y a su vez también se puede observar para Checoslovaquia. Para el caso de Alemania y Austria, con un mayo o menor rezago, se observa que el componente que toma relevancia luego de la caída del componente 5, es el 11 (Industria de media tecnología, automóviles y maquinaria). Por su parte, para el caso de República Checa, junto con el componente 11, también toma igual importancia el 10, que también es industrial con un fuerte componente de automóviles y maquinaria, pero con mayor peso en productos electrónicos, como cables, paneles de control, televisores, etc. 

\begin{figure}
	\centering
	\includegraphics[width=0.7\linewidth]{../desagregado/LDA/graficos/graficoLDA_k30_AUT_CZE_DEU}
	\caption{Distribución componentes: Austria, República Checa y Alemania.}
	\label{fig:graficoldak30autczedeu}
\end{figure}


En la figura \ref{fig:graficoldak30dnkfinnldswe} se presentan los resultados para algunos de los países nórdicos: Dinamarca, Finlandia, los Países Bajos y Suecia. Si bien para todos ellos, excepto Finlandia, se puede observar también la caída del componente 5, el mismo no ocupó en ningún momento de la serie el lugar del componente más importante. Al contrario, Todos estos países presentan una especialización bien definida, aunque no sin cambios en el tiempo. En los casos de los países escandinavos, Finlandia y Suecia, predomina el componente 17 (Industria y Agro. Motores, madera y derivados), aunque con una tendencia decreciente en el tiempo. Para el caso finlandés, crece desde mediados de los 90' el componente 23 (procesadores y microcircuitos) y el componente 11(automóviles y maquinaria) a partir de los 2000. El caso sueco es similar, aunque el componente 11 crece tendencialmente desde los 80' y su peso es en general mayor. Tanto el el rol tradicional de estos países como productores de materias primas, y en particular madera y derivados, como los recientes procesos de complejización productiva son reconocidos en la literatura \citep{blomstrom2003natural}. En el caso de Dinamarca y los Países Bajos el componente característico es el 29 (Industria y Agro, maquinaria,productos químicos, flores, quesos, etc.). En el caso de Dinamarca, también se observa al principio de la serie un peso importante en el componente 24 (carnes y lácteos), que cae tendencialmente. Por su parte también se observa un peso importante del componente 2 (industria liviana), particularmente a partir de la mencionada caída del componente 5. 

\begin{figure}
	\centering
	\includegraphics[width=0.7\linewidth]{../desagregado/LDA/graficos/graficoLDA_k30_DNK_FIN_NLD_SWE}
	\caption{Distribución componentes: Dinamárca, Finlandia, los Países bajos y Suecia.}
	\label{fig:graficoldak30dnkfinnldswe}
\end{figure}

%\begin{figure}
%	\centering
%	\includegraphics[width=0.7\linewidth]{../desagregado/LDA/graficos/graficoLDA_k30_BEL_LUX}
%	\caption{Distribución componentes: Bélgica y Luxemburgo.}
%	\label{fig:graficoldak30bellux}
%\end{figure}

En la figura \ref{fig:graficoldak30fragbrirl} se presenta la evolución de las exportaciones de Francia, Irlanda y el Reino Unido. Nuevamente se observa la caída del componente 5 entre 1977 y 1978, aunque en el caso de Irlanda este componente tenía un peso menor, aunque creciente. Las exportaciones de Irlanda al principio de la serie se caracterizaban por el componente 24 (carnes, lácteos, etc.). Si bien el componente 5 era predominante al comienzo de la serie en Francia y Reino Unido, tal como en los mencionados casos de Alemania y Austria, luego del punto de inflexión cada uno de estos países tiene un componente característico diferente: En el caso francés, es el 26 (aviones,motores, perfumería, vino y queso), todos ellos productos típicos, inclusive la producción de aviones, donde este país es uno de los líderes mundiales con la empresa Air Bus a la cabeza de dicho sector \citep{frenken2000complexity}. En el caso del Reino Unido el componente 30 (motores, medicamentos y bebidas alcohólicas) es el que toma la delantera. En ambos casos, estos componentes ya tenían un valor significativo al comienzo de la serie. Distinto es el caso de Irlanda, en donde el componente 31 (medicamentos, químicos y, en menor medida, maquinaria digital) no tenía relevancia alguna al comienzo de la serie. Este cambio desde un componente basado en productos primarios a uno de alta complejidad es reconocido en la literatura bajo el nombre de \textit{celtic tiger}, en relación al proceso de industrialización reciente en el sudeste asiático (\textit{asian tigers}) \citep{murphy2000celtic}.

\begin{figure}
	\centering
	\includegraphics[width=0.7\linewidth]{../desagregado/LDA/graficos/graficoLDA_k30_FRA_GBR_IRL}
	\caption{Francia, Reino Unido e Irlanda.}
	\label{fig:graficoldak30fragbrirl}
\end{figure}

La figura \ref{fig:graficoldak30espitaprt} muestra la evolución de Italia, Portugal y España. En el caso italiano, nuevamente se observa una caída del componente 5, que es reemplazado por el 2 (calzado, muebles y maquinaria), en términos de la clasificación de Lall, este cambio implica una simplificación productiva, dado que el peso de productos de alta tecnología en el componente 2 es menor que en el 5. El caso de España y Portugal es similar: desde el comienzo de la serie el componente dominante es el 18 (frutas y vegetales, y producción textil), aunque su peso disminuye a lo largo de la serie, lo cual se contrapone con el incremento de la importancia del componente 3 (automóviles, polietileno, hierro y acero). En particular para el caso español, a partir de la década del 90' este último componente es de mayor peso. Este proceso de industrialización tardía de España y Portugal es reconocido en la literatura \citep{holman2005integrating}.

\begin{figure}
	\centering
	\includegraphics[width=0.7\linewidth]{../desagregado/LDA/graficos/graficoLDA_k30_ESP_ITA_PRT}
	\caption{Distribución componentes: España, Italia y Portugal.}
	\label{fig:graficoldak30espitaprt}
\end{figure}

La caracterización del comercio de los países europeos deja como conclusión por un lado la particularidad del comercio dentro del espacio de influencia de la unión soviética, así como la similitud de su estructura productiva en la actualidad de los países que lo componían. Por su parte, es marcada la regularidad temporal con la que el componente 5 cae en importancia en Austria, Alemania, Francia, Reino Unido e Italia. A su vez, parece notarse un proceso de especialización y complementariedad productiva al interior de lo que actualmente constituye la Unión Europea, dado que mientras al principio de la serie muchos de estos países producían en un mismo componente, la caída del mismo da lugar a componentes más o menos bien diferenciados entre los países. También se observa la industrialización tardía de España, Portugal e Irlanda, y se logra apreciar la similitud de este proceso en los primeros dos países, y sus diferencias, tanto en magnitud como en especialización productiva, con Irlanda. 


\paragraph{Asia}

en la figura \ref{fig:graficoldak30chnjpnphl} se puede observar la evolución de las exportaciones de China, japón y Filipinas. En el caso de China, a comienzos de los 60' el componente más relevante es el 28, compuesto por arroz, algodón, té y algunos productos textiles. Este componente tiene una caída tendencial, acompañada de un incremento del componente 4 (vestimenta, juguetes, etc.) que a partir de la década del 80' es el de mayor peso. Este componente tiene su pico a comienzos de los 90', para comenzar a descender, en sintonía con el crecimiento del componente 23 (televisores,computadoras, microcircuitos y transistores), que constituye en la actualidad aproximadamente el 80\% de las exportaciones de este país. Este cambio en la especificidad productiva da cuenta de la complejización de la industria China por etapas, primero desde una economía básicamente agropecuaria a una industrialización de baja complejidad, para luego convertirse en uno de los principales exportadores mundiales de productos de alta complejidad \cite{chenery1986industrialization, costantino2013gatopardismo}. 
En el caso de Japón, se observa al igual que en muchos países europeos la importancia del componente 5 al comienzo de la serie. Sin embargo, la evolución es distinta en este caso. En primer lugar, si bien hay una caída pronunciada en 1977, no tiene la misma magnitud que en los casos europeo o de Estados Unidos. A su vez, el componente 6 (motores, barcos y maquinaria eléctrica) cobra importancia desde el comienzo de la serie, y supera al 5 ya desde comienzos de la década del 70'. A su vez, desde mediados de los 80' aparece el componente 27 (microcircuitos, unidades periféricas, transistores, etc.). 
El caso de las Filipinas es paradigmáticamente distinto al de China. Desde comienzos de la serie hasta los 90', el componente principal es el 9 (Café, bananas, etc.). Este componente cae de forma irregular hasta 1990. y a partir de mediados de los 90' el componente 27 crece drásticamente \citep{chang2018industrial}.

\begin{figure}
	\centering
	\includegraphics[width=0.7\linewidth]{../desagregado/LDA/graficos/graficoLDA_k30_CHN_JPN_PHL}
	\caption{Distribución componentes: China, Japón y Filipinas.}
	\label{fig:graficoldak30chnjpnphl}
\end{figure}


La figura \ref{fig:graficoldak30hkgkorsgptwn} muestra la evolución de los denominados \textit{tigres asiáticos}: Hong-Kong, Singapur, Corea del sur y Taiwan. 
En el caso de Hong-Kong, el componente cuatro (calzado, ropa y juguetes) es el de mayo relevancia a lo largo dela serie, hasta el cambio de siglo. A partir de allí, y de forma irregular, disminuye su importancia respecto del componente 27 (microcircuitos, transistores, etc.). Un comportamiento similar tienen Taiwan y Corea del Sur, donde al comienzo de la serie tiene mayor peso el componente 28 (arroz, algodón, etc.), que pierde importancia a mediados de los 60' para ser remplazado por el componente 4, y por el 27 a mediados de los 90'. Vale mencionar que el componente 28 también tiene un peso significativo y decreciente al comienzo de la serie de Hong-Kong, lo que da la pauta de que esta región podría haber atravesado un proceso similar al de los demás países, con anterioridad en el tiempo. Estos procesos son similares a su vez con lo analizado para el caso Chino, donde también se observa el proceso de sofisticación productiva de los componentes 28-4-27. En el caso de Corea del sur, junto con el componente 27, también crece en importancia el 6 (motores, barcos y maquinaria eléctrica), visto en el caso japonés. 
Singapur tiene un recorrido diferente al de los demás \textit{tigres}. Hasta fines de los 80' el componente más importante es el 20 (fuel oil, combustibles, etc.), para ser reemplazado por el 27, que se encontraba en rápido crecimiento desde principios de la década. Siguiendo a \cite{ng2013singapore} podría decirse que la industrialización de Singapur se basa en su producción de energía. Sin embargo, a diferencia de los demás países, también se observa en una caída del componente 27 a partir de los 2000, y un crecimiento del 14 (derivados del petróleo)

\begin{figure}
	\centering
	\includegraphics[width=0.7\linewidth]{../desagregado/LDA/graficos/graficoLDA_k30_HKG_KOR_SGP_TWN}
	\caption{Distribución componentes: Hong Kong, Corea del Sur, Singapur y Taiwan.}
	\label{fig:graficoldak30hkgkorsgptwn}
\end{figure}

Finalmente, en el gráfico \ref{fig:graficoldak30bgdindpak} se presenta el caso de Bangladesh, India y Pakistán. Estos tres países tienen un recorrido muy diferente al visto en el resto de Asia. En el caso de Pakistán, el componente 28 representa la absoluta mayoría de las exportaciones, con un decrecimiento tendencial muy leve a partir de los 90' y un incremento muy pequeño de los componentes 4 (ropa juguetes, etc.) y 18 (productos primarios y textiles). En el caso de la India, el componente 28 es predominante, pero su tendencia descendente es marcada, y se comporta en espejo con el crecimiento del componente 15 (diamantes). Este último se explica no por las escasas minas de diamantes que existen en la actualidad en la India, sino por la industria de la talla de diamantes en el oeste de dicho país \citep{henn2013transnational,sevdermish1998rise}. 
Por su parte, Bangladesh muestra un recorrido similar al de China, pasando del componente 28 al 4. 


\begin{figure}
	\centering
	\includegraphics[width=0.7\linewidth]{../desagregado/LDA/graficos/graficoLDA_k30_BGD_IND_PAK}
	\caption{Distribución componentes: Bangladesh, India y Pakistán.}
	\label{fig:graficoldak30bgdindpak}
\end{figure}

La evolución de las exportaciones en Asia deja un imagen muy diferente a la de Europa. Mientras en aquel continente se pasa de una concentración de los distintos países en un único componente a una diferenciación, en el caso asiático sucede lo contrario: ocurre un proceso de indiferenciación en las exportaciones, concentrándose en la producción de industria liviana de baja complejidad primero, y la producción de micro-componentes luego. A su vez, lo que se observa es que la producción del hardware asociado a las nuevas tecnologías que se producen a partir del cambio técnico de la \textit{era digital}, y que generan la obsolescencia de los productos comprendidos en el componente 5, se producen en el sudeste asiático.  


\subsection{Grafo Bipartito}

\subsubsection{Proyección a nivel países}

Como se mencionó en la metodología, se realizó una proyección del grafo bipartito para reconstruir un grafo simple ponderado de la similitud de la estructura productiva de los países. En esta red, por lo tanto, la cercanía entre dos países esta determinada por cuán similares son sus canastas exportadoras. Sobre la base de esto, se realizó un análisis de comunidades. El objetivo de este ejercicio es encontrar nueva evidencia empírica respecto de un debate difundido en la literatura económica sobre la Nueva División Internacional del trabajo \cite{starosta2016revisiting}. Por ello, en la figura \ref{fig:mapas_proyeccion_louvain} se representan las comunidades resultantes del clustering de Louvain\footnote{En el anexo, sección \ref{sec:fig_comp} se presentan los resultados utilizando el método de walktrap.}, en mapas de diferentes épocas. La figura \ref{fig:mapas_proyeccion_1} muestra el resultado para la proyección en 1966. en este período el mundo se encuentra fundamentalmente dividido en dos clusters:

\begin{itemize}
	\item La primera comunidad esta constituida por Estados Unidos, Europa occidental y algunso países del sudeste asiático, como Japón, China y Corea del Sur. 
	\item La segunda comunidad esta compuesta por prácticamente la totalidad del resto del planeta, con excepción de unos pocos países de Centroamérica. 
\end{itemize}

Esta división se podría caracterizar como la diferenciación entre países productores de mercancías industriales de diverso tipo, respecto de países productores de otras mercancías, de origen agropecuario o minero. Esta imagen de un mundo polarizado es coincidente con la bibliografía respecto de la División Internacional del Trabajo clásica, en la cual el mundo se caracterizaba como dividido entre \textit{centro} y \textit{periferia}, donde los primeros son productores de mercancías industriales, mientras que los segundo son productores de materias primas \citep{coffey1996newer,hutchinson2004globalisation}. 

En la figura \ref{fig:mapas_proyeccion_2} esta imagen empieza a cambiar. Si bien existe aún un dos clusters que coincidentes con los caracterizados para el decenio anterior, surge una nueva comunidad, de países del continente asiático como China e India y Pakistán, entre otros, junto con países africanos y de medio oriente con un cierto grado de desarrollo como Turquía Marruecos y Egipto, y algunos países americanos como México y Colombia. Este cluster, disperso geográficamente, puede caracterizarse como la comunidad de países cuya producción es industrial, de baja complejidad. Esta comunidad es coincidente en tiempo y espacio con lo propuesto por la teoría de la Nueva División Internacional del trabajo, según la cual el fenómeno que ocurre durante este decenio es la fragmentación del proceso productivo y la transferencia de las etapas simples de la producción industrial a estos países, en busca de la mano de obra disponible \citep{frobel1978new, haggard1986newly}. A partir de este punto, lo que se observa para los siguientes decenios es que se conserva la estructura de países productores de mercancías industriales de alta complejidad, baja complejidad y de materias primas.
En la figura \ref{fig:mapas_proyeccion_3} se puede observar la distribución de las comunidades en el año 1986. Los cambios en este punto no son de tipo \textit{estructural}, sino que implican movimientos entre las comunidades anteriormente descriptas. En la década de los 80' el principal cambio se observa en América, en dónde tanto Canadá como Brasil se incorporan en la primera comunidad, mientras que México parecería complejizar su producción industrial \citep{cardoso2009brief, feenstra1997foreign}.
En la figura \ref{fig:mapas_proyeccion_4} se incorpora la información respecto del comercio de la ex Unión Soviética, que se incorpora al mercado mundial como productora de materias primas en el caso de Rusia y los países de Asia central y como exportadora de productos industriales en el caso de los países de Europa oriental. A su vez, se observa como nuevos países del Sudeste asiático se incorporan a la comunidad de países productores de mercancías de baja complejidad, cómo en el caso de Indonesia, Laos, Vietnam, mientras que países como Corea del Sur, Taiwan y Malasia complejizan su producción, este proceso es reconocido en la literatura como la \textit{segunda ola} de los tigres asiáticos \citep{rodrik1995getting}.
Finalmente, en la figura \ref{fig:mapas_proyeccion_5} se puede observar la configuración de las comunidades en el año 2016. Allí se observa un proceso de reprimarización de Latinoamérica, dado que por primera vez desde 1966 en subcontinente en pleno se encuentra en la comunidad de productores de materias primas \citep{bolinaga2015consenso,svampa2015commodities}. Por su parte, también se observa como continúa la complejización de la producción en el sudeste asiático, con la incorporación de China, Tailandia y Filipinas a la primera comunidad \citep{suehiro2008catch}. 


\begin{figure}
	\centering
	\subfigure[1966]{\label{fig:mapas_proyeccion_1}\includegraphics[width=0.45\linewidth]{mapa_projection_louvain_lp_1966}}
	\subfigure[1976]{\label{fig:mapas_proyeccion_2}\includegraphics[width=0.45\linewidth]{mapa_projection_louvain_lp_1976}}
	\subfigure[1986]{\label{fig:mapas_proyeccion_3}\includegraphics[width=0.45\linewidth]{mapa_projection_louvain_lp_1986}}
	\subfigure[1996]{\label{fig:mapas_proyeccion_4}\includegraphics[width=0.45\linewidth]{mapa_projection_louvain_lp_1996}}
	\subfigure[2016]{\label{fig:mapas_proyeccion_5}\includegraphics[width=0.5\linewidth]{mapa_projection_louvain_lp_2016}}
	\caption{Proyección a países del grafo bipartito. Clustering Louvain. Exportaciones}
	\label{fig:mapas_proyeccion_louvain}
\end{figure}

Por comparación y de forma complementaria a los resultados anteriores, en la figura \ref{fig:mapas_comercio} se presentan los resultados de la misma metodología, aplicada sobre los datos de comercio agregado entre países del capítulo dos\footnote{En el anexo, sección \ref{sec:fig_comp} se presentan los resultados utilizando el método de Walktrap}. En primer lugar lo que destaca es la mayor cantidad de comunidades, así como el número cambiante en el tiempo. Esto puede interpretarse como que las relaciones comerciales, bilaterales o multilaterales, son más inestables en el tiempo que el rol que los países juegan en el mercado mundial en tanto productores de cierto tipo específico de productos. En otros términos, con quién se comercia es una forma concreta circunstancial del verdadero contenido puesto en juego en dicho comercio, que es el rol de los países en el mercado mundial \citep{starosta2016revisiting}. En segundo lugar, lo que se observa es que las distancias geográficas parecieran tener un rol más relevante en el caso de las comunidades comerciales, respecto de las comunidades según estructura productiva. Esto es esperable, dado los costos involucrados en el transporte de mercancías \citep{Head2014}. En todas las figuras, con excepción de \ref{fig:mapas_comercio_2}, existe una comunidad de América. A su vez, en todas las figuras hasta 2016 se puede observar el efecto de la Unión Soviética sobre los vínculos comerciales \footnote{Vale recordar en este punto que, mientras los datos hasta el 2000 provienen de \cite{Gleditsch2002}, los datos de 2016 fueron elaborados a partir de la base de datos de  \textit{comtrade}.} en la actualidad, representada en la figura \ref{fig:mapas_comercio_5}, lo que se observa es una comunidad comercial de Europa y Asia Central, con excepción del Reino Unido. También se observa una comunidad en África austral, central y oriental, mientras que África occidental constituye una comunidad propia, a la cual pertenecen también parte del sudeste asiático y Oceanía.


\begin{figure}
	\centering
	\subfigure[1966]{\label{fig:mapas_comercio_1}\includegraphics[width=0.45\linewidth]{mapa_louvain_lp_1966}}
	\subfigure[1976]{\label{fig:mapas_comercio_2}\includegraphics[width=0.45\linewidth]{mapa_louvain_lp_1976}}
	\subfigure[1986]{\label{fig:mapas_comercio_3}\includegraphics[width=0.45\linewidth]{mapa_louvain_lp_1986}}
	\subfigure[1996]{\label{fig:mapas_comercio_4}\includegraphics[width=0.45\linewidth]{mapa_louvain_lp_1996}}
	\subfigure[2016]{\label{fig:mapas_comercio_5}\includegraphics[width=0.5\linewidth]{mapa_louvain_2016}}
	\caption{Grafo a nivel países. Clustering Louvain. Exportaciones}
	\label{fig:mapas_comercio}
\end{figure}





%En la figura \ref{fig:mapas_proyeccion} se puede observar las comunidades detectadas para el año 2016. En \ref{fig:mapas_proyeccion_1} se puede observar las comunidades obtenidas mediante la técnica de \textit{Louvain}. Con este método se encuentras tres comunidades. La primer comunidad contiene a China, India, varios países del Sudeste asiático, y algunos países de África y el Caribe. La segunda comunidad contiene al continente sudamericano, Australia, Nueva Zelanda, la mayoría del continente africano, Rusia y oriente medio. Finalmente, el tercer cluster contiene a América del norte, Europa occidental, con excepción de Portugal, Japón, Corea del Sur, Malasia y Tailandia. Estos grupos pueden ser caracterizados respectivamente como:
%
%\begin{enumerate}
%	\item Productores de industria de baja complejidad, textiles, etc.
%	\item Productores de materias primas.
%	\item Productores de industria de alta complejidad.
%\end{enumerate}
%
%En la figura \ref{fig:mapas_proyeccion_2} se presentan los grupos obtenidas utilizando la detección de comunidades de \textit{Walktrap}. Allí lo primero que destaca es que los países petroleros, como Rusia y Arabia Saudita pasaron al tercer cluster. Es decir que esta comunidad sería la de países de industria de alta complejidad y petroleros. Por su parte India pasa a formar parte del cluster de productores de materias primas, y Nueva Zelanda al de países productores de industria de alta complejidad. Una particularidad es la República Centroafricana, que se encuentra en el cluster de productos de alta complejidad. Esto último se explica por el hecho de que en 2012 comienza una guerra civil en dicho país que implica un comercio internacional muy bajo y volátil, y en particular por algún motivo para dicho año este país cuenta entre sus productos más exportados, tanques de guerra y demás vehículos de combate.
%
%Los países petroleros tienen una especificidad particular y les correspondería una comunidad propia, y posiblemente dado que son pocos países ninguno de los métodos utilizados logra dar con esta especificidad. Independientemente de estas particularidades, los resultados obtenidos parecieran confirmar la literatura respecto al rol de los países en el mercado mundial en los últimos años \citep{frobel1978new}.

%\begin{figure}
%	\centering
%	\subfigure[Cluster Louvain]{\label{fig:mapas_proyeccion_1}\includegraphics[width=\linewidth]{mapa_projection_louvain}}
%	\subfigure[Cluster Walktrap]{\label{fig:mapas_proyeccion_2}\includegraphics[width=\linewidth]{mapa_projection_walktrap}}
%	\caption{Proyección países. Exportaciones 2016}
%	\label{fig:mapas_proyeccion}
%\end{figure}

%Para comparar los resultados obtenidos, en la figura \ref{fig:mapa_comercio_agregado} se muestra los clusters de Louvain para el comercio agregado a nivel países (ver capítulo \ref{sec:agregado}). La imagen que se obtiene de las comunidades es muy distinta:
%
%\begin{itemize}
%	\item La primera comunidad esta compuesta por el contiente americano y Mongolia.
%	\item La segunda comunidad esta compuesta por medio oriente, excepto Israel, y parte de África del norte.
%	\item A la tercer comunidad pertenecen Europa, excepto el Reino Unido, Bélgica y Portugual.
%	\item El cuarto cluster se corresponde con el sudeste asiático y África noroccidental.
%	\item La quinta comunidad se corresponde con el África austral.
%	\item Aunque difícil de reconocer en el mapa, el último cluster esta compuesto por Palau, Andorra, Guam y Micronesia.
%\end{itemize}
%
%Aquí se pueden apreciar los regionalismos comerciales, y el rol que la distancia geográfica juega en el comercio internacional \citep{Head2014}. De las figuras \ref{fig:mapas_proyeccion} y \ref{fig:mapa_comercio_agregado} se puede concluir que las distintas representaciones de la información del comercio internacional, según socios comerciales o similitud en la estructura productiva, logran dar con distintas expresiones del sistema económico mundial, siendo todas estas parte de un mismo fenómeno.
%
%\begin{figure}
%	\centering	
%	\includegraphics[width=\linewidth]{mapa_louvain}
%	\caption{Cluster Louvain. Comercio agregado. Exportaciones 2016}
%	\label{fig:mapa_comercio_agregado}
%\end{figure}

\subsubsection{Espacio de productos} \label{sec:espacio_productos}

Como se mencionó en la metodología, para analizar las relaciones entre los productos se toma como base a \cite{Hidalgo2009} y su concepto de \textit{Espacio de productos}. el mismo se define a partir de un concepto de distancia, definido como \textit{proximidad} analizado en la sección correspondiente de la metodología. Esto se aplico sobre el RCA promedio del período 1996-2017, para SITC desagregado a 4 dígitos. Para realizar una primera caracterización del espacio de productos, en \ref{fig:proximity} se muestra el mapa de calor. En \ref{fig:proximity_1} los productos se encuentran ordenados alfabéticamente según su nomenclador SITC. En \ref{fig:proximity_2} la matriz se ordena a partir de realizar un clustering jerárquico. Dado que algunos pares de productos tienen una proximidad igual a cero, al calcular la matriz de distancias estos pares tienen distancia infinita, por lo que la misma se reemplaza por un valor alto fuera de rango.

Más allá de este ajuste para realizar el clustering, en \ref{fig:proximity_2} se mantienen los valores originales para todos los $\phi_{ij}$. Las diferencias que se observan entre ambas figuras dan cuenta que el nomenclador SITC no mantiene el ordenamiento propio del espacio de productos dada la similitud en términos de proximidad. Este resultado refuerza la motivación del nomenclador construido a partir de \textit{Latent Dirichlet Allocation Models} propuesto en la sección precedente. 

\begin{figure}
	\centering
	\subfigure[Ordenamiento SITC]{\label{fig:proximity_1}\includegraphics[width=.45\linewidth]{heatmap_prox_sitcOrd}}
	\subfigure[Ordenamiento Clustering jerárquico]{\label{fig:proximity_2}\includegraphics[width=.45\linewidth]{heatmap_prox_ClustOrd}}
	\caption{Heatmap proximidad. Espacio de productos. Exportaciones. Promedio 1996-2017}
	\label{fig:proximity}
\end{figure}

En ambos mapas de calor se puede observar que existe un pequeño conjunto de productos de muy alta proximidad. Para observar dicho fenómeno, en la tabla \ref{table:similarity} se destacan los diez pares productos de mayor proximidad, y su descripción a cuatro dígitos. Como se observa, todos ellos son productos textiles, algunos de ellos incluso compartiendo descripción a cuatro dígitos. Los procesos productivos para realizar estas mercancías son muy similares y requieren básicamente de los mismos insumos, maquinaria y mano de obra, lo cual explica su similitud. 


\begin{table}[]
	\begin{tabular}{|r|l|l|l|l|}
		\hline
		\textbf{Sim}               & $SITC_1$ & \textbf{Descripción}                                                                                                     & $SITC_2$ & \textbf{Descripción}                                                                                                     \\ \hline
		0.82                       & 8438     & underwear,nightwear etc.                                                                                                 & 8448     & underwear, nightwear etc                                                                                                 \\ \hline
		0.82                       & 8425     & \begin{tabular}[c]{@{}l@{}}skirts and divided skirts,\\ women's or girls',\\ of textile materials, not k\end{tabular}    & 8427     & \begin{tabular}[c]{@{}l@{}}blouses,\\ shirts and shirt-blouses,\\ women's or girls', \\ of textile material\end{tabular} \\ \hline
		0.82                       & 8416     & underwear,nightwear etc.                                                                                                 & 8428     & underwear,nightwear etc.                                                                                                 \\ \hline
		0.81                       & 8424     & \begin{tabular}[c]{@{}l@{}}dresses,\\ women's or girls',\\ of textile materials,\\ notted or crochete\end{tabular}       & 8425     & \begin{tabular}[c]{@{}l@{}}skirts and divided skirts,\\ women's or girls',\\ of textile materials, not k\end{tabular}    \\ \hline
		0.80                       & 8427     & \begin{tabular}[c]{@{}l@{}}blouses,\\ shirts and shirt-blouses,\\ women's or girls', \\ of textile material\end{tabular} & 8424     & \begin{tabular}[c]{@{}l@{}}dresses,\\ women's or girls',\\ of textile materials,\\ notted or crochete\end{tabular}       \\ \hline
		0.80                       & 8425     & shirts and shirt-blouses,                                                                                                & 8426     & \begin{tabular}[c]{@{}l@{}}trousers, \\ bib and brace overalls,\\ breeches and shorts,\\ women's or girls\end{tabular}   \\ \hline
		0.80                       & 8428     & women's or girls',                                                                                                       & 8448     & underwear, nightwear etc                                                                                                 \\ \hline
		0.79                       & 8411     & of textile material                                                                                                      & 8421     & overcoats,oth.coats etc.                                                                                                 \\ \hline
		0.79                       & 8415     & shirts                                                                                                                   & 8426     & \begin{tabular}[c]{@{}l@{}}trousers, \\ bib and brace overalls,\\ breeches and shorts,\\ women's or girls\end{tabular}   \\ \hline
		\multicolumn{1}{|l|}{0.79} & 8426     & \begin{tabular}[c]{@{}l@{}}trousers, \\ bib and brace overalls,\\ breeches and shorts,\\ women's or girls\end{tabular}   & 8427     & \begin{tabular}[c]{@{}l@{}}blouses,\\ shirts and shirt-blouses,\\ women's or girls', \\ of textile material\end{tabular} \\ \hline
	\end{tabular}
\caption{Top 10 pares de productos más similares}
\label{table:similarity}
\end{table}


Teniendo presente lo anterior, se utilizó la inversa de la matriz de similaridad como una matriz de distancias para calcular clusters con mediante el método \textit{Partition around medioids}. A su vez, con estos datos se reconstruyó el grafo de productos utilizando una versión modificada de la matriz de proximidad como matriz de adyacencia del grafo. Específicamente, para aquellos productos con una proximidad menor a 0.5 se anuló la métrica: 


$$
\phi_{ij} = 
\begin{cases} 
\phi_{ij} & si \ \phi_{ij}>0.5 \\
0 & si \ \phi_{ij}<0.5
\end{cases}
$$

Con esta matriz de proximidad modificada se reconstruyó el grafo. En la figura \ref{fig:clustering2} se observa el componente gigante, identificando la pertenencia de cada nodo y su medioide.


\begin{figure}
	\centering	
	\includegraphics[width=0.7\linewidth]{pam2_gigant}
	\caption{Componente gigante. Grafo de proximidad y clustering por K-medioides. Exportaciones. Promedio 1996-2017. 2 medioides}
	\label{fig:clustering2}
\end{figure}

En la figura \ref{fig:clustering2} se puede observar una buena separación entre ambos clusters dentro del componente gigante espacio de productos. En la tabla \ref{table:pam2} se describen los medioides. Dado que el \textit{Aldehyde} es un químico para sintetizar otros compuestos, mientras que el otro mediodide esta caracterizado por muebles de madera, podemos decir que el cluster en verde representa productos de alta complejidad, mientras que el cluster rojo representa a los productos de baja complejidad. 

\begin{table}
	\centering
	\begin{tabular}{ll}
		\hline
		medioide & Description \\ 
		\hline
		8215 & Furniture,nes,of wood \\ 
		5162 & Aldehyde,etc.fnct.cmpnds \\ 
		\hline
	\end{tabular}
	\caption{Medioides. PAM. k=2} 
	\label{table:pam2}
\end{table}


En la figura \ref{fig:clustering10} podemos observar qué sucede con 10 medioides. Aquí, si bien se mantiene una separación de los grupos en muchos casos, podemos observar que existe un solapamiento en los puntos más densos del componente gigante. Estos clusters representan a productos de alta complejidad, como componentes de maquinarias (7285), componentes químicos (5416), o productos electrónicos. Con mejor separación aparece la industria pesada (6911) y en especial equipos para el procesamiento automático de datos (7526). En la figura \ref{table:pam10} se describen los demás medioides.

\begin{figure}
	\centering	
	\includegraphics[width=0.7\linewidth]{pam10_gigant}
	\caption{Componente gigante. Grafo de proximidad y clustering por K-medioides. Exportaciones. Promedio 1996-2017. 10 medioides}
	\label{fig:clustering10}
\end{figure}


\begin{table}
	\centering
	\begin{tabular}{ll}
		\hline
		medioide & Description \\ 
		\hline
		6911 & Metal structures,parts \\ 
		2831 & Copper ores and concentrates \\ 
		6954 & Hand tools,etc. nes \\ 
		2232 & Palm nuts and kernels \\ 
		8451 & Babies'garmnts,clths acc \\ 
		8121 & Boilrs.radiatrs,etc.n.el \\ 
		5416 & Glycosides; glands etc. \\ 
		6572 & Non-wovens, whether or not impregnated, coated, covered or laminated, n.e \\ 
		7526 & Input or output units for automatic data processing machines, whether or \\ 
		7285 & Parts publc wrk mach etc \\ 
		\hline
	\end{tabular}
	\caption{Medioides. PAM. k=10} 
	\label{table:pam10}
\end{table}


En la figura \ref{fig:clustering50} se analiza el caso de 50 medioides. La descripción de cada uno de ellos se encuentra en el anexo, en la tabla \ref{tabla:pam50}. Allí se fuerza lo mencionado anteriormente sobre el solapamiento en la parte más densa del grafo, siendo casi todos los medioides productos del capítulo 7, de maquinaria y equipos industriales. Es decir, es un sector del grafo de alta complejidad. Se mantiene como un cluster más o menos bien diferenciado los procesadores automáticos de datos y microcomponentes (7526, 7722), el cual se podría caracterizar como un sector de mayor complejidad que el centro del grafo. También se configura separadamente el sector de productos textiles (6562) y el de herramientas manuales (6954). 

\begin{figure}
	\centering	
	\includegraphics[width=\linewidth]{pam50_gigant}
	\caption{Componente gigante. Grafo de proximidad y clustering por K-medioides. Exportaciones. Promedio 1996-2017. 50 medioides}
	\label{fig:clustering50}
\end{figure}



\subsection{Conclusiones}

En el presente capítulo se realizó una segunda aproximación a la cuantificación del comercio internacional, considerando la dimensión \textit{producto}. Este enfoque busca no considerar el lugar que ocupan los países en el mercado mundial, no ya desde sus centralidad en el comercio agregado, sino en función del rol de los mismos en el sistema productivo mundial. Para ello, y dado que parte del presente capítulo fue elaborada en un marco de trabajo multidisciplinario\footnote{Parte del trabajo del presente capítulo se realizó en el marco  del proyecto de Investigación Científica y Tecnológica (PICT) 1085-2016. 'Abordando la restricción externa en América Latina a partir de la integración regional: integración productiva, cooperación Sur-Sur y financiamiento para el desarrollo'.}, se desarrollaron diversas herramientas de análisis. En primer lugar, para poder dar a conocer los resultados a investigadores temáticos, se elaboraron diversas aplicaciones web en \textit{shiny}, tanto para el análisis exploratorio de datos \footnote{\url{ https://treemaps.shinyapps.io/treemaps/ }}, como para la caracterización de los componentes del modelo de \textit{LDA}, y sus resultados \footnote{\url{ https://treemaps.shinyapps.io/LDA_worldtrade/ }}. La propuesta metodológica del capítulo consta de dos partes. En primer lugar, se propone una nueva utilización del modelo propuesto por \cite{blei2003latent}, proveniente del campo del análisis de textos, para la detección de las dimensiones latentes del comercio mundial, nombradas \textit{componentes}. Allí se exploró los parámetros óptimos del modelo, se caracterizaron los componentes, y con esto, se analizaron los resultados desde el punto de vista de la evolución de la estructura productiva exportadora de los países a lo largo del período 1962-2016. Esta nueva herramienta se presenta como promisoria para la elaboración de una caracterización sintética del comercio de los países, sin necesidad de elaborar nomencladores ad-hoc.
Por su parte, tomando el trabajo de  \cite{balassa1965trade} se elaboró un grafo bipartito entre países y productos, con el que luego se elaboró una proyección al grafo de países y de allí se caracterizaron las comunidades presentes en el mercado mundial a lo largo del período. Este análisis permitió agrupar a los países según el rol que los mismos juegan en la producción mundial de valor. De esta forma, se produjo nueva evidencia empírica respecto de la Nueva División Internacional del trabajo. Finalmente, sobre la base del trabajo de \cite{Hidalgo2009} se caracterizó el espacio de productos, en función de su similitud. Śe encontró de esta forma que los nomencladores oficiales no son una representación uniforme respecto a la distancia entre los distintos productos, y que por lo tanto resultan de interés los ejercicios como el propuesto en utilizando LDA. A su vez, mediante técnicas de clustering, se caracterizó a los productos procurando construir tipologías basadas en la información empírica disponible.
En síntesis, la propuesta del presente capítulo es elaborar metodologías complementarias a las tradicionalmente utilizadas en el análisis estadístico del comercio internacional. Las diferentes propuestas tienen en común buscar tipologías que se basen en los datos generados por el comercio internacional.


%\bibliographystyle{unsrt}
%\bibliography{bibliografia}
%
\end{document}

