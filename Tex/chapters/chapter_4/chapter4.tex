%\documentclass[../../tesis.tex]{subfiles}
\documentclass[class=article, crop=false]{standalone}
\usepackage[subpreambles=true]{standalone}
\usepackage{import}
\graphicspath{{images/}}


%
%
%
%
\usepackage{amssymb}
\usepackage{amsmath}
%\usepackage{natbib}
\setcounter{tocdepth}{3}
\usepackage{graphicx}
%\graphicspath{ {Graficos/} }
\usepackage{subfigure}
\usepackage{gensymb}
\usepackage{authblk}
\usepackage{url}
\usepackage[utf8]{inputenc}

\usepackage[spanish]{babel}
\selectlanguage{spanish}
%\usepackage[style=authoryear]{biblatex}

%
%\newcommand{\keywords}[1]{\par\addvspace\baselineskip
%\noindent\keywordname\enspace\ignorespaces#1}
%\renewcommand\keywordname{Palabras Clave:}

\begin{document}


\pendiente{Si la representación del capítulo tercero generaba la impresión de que se trata de un fenómeno en el cual elementos aislados, países, se interrelacionan, el capítulo cuarto resuelve dicha impresión dando la pauta de que el sistema económico funciona a nivel de sistema-mundo, y en donde los países son partes de una organización más general, que en la literatura se describe como la  Nueva División Internacional del Trabajo}



\end{document}

