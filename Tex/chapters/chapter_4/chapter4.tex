%\documentclass[../../tesis.tex]{subfiles}
\documentclass[class=article, crop=false]{standalone}
\usepackage[subpreambles=true]{standalone}
\usepackage{import}
\graphicspath{{images/}}


%
%
%
%
\usepackage{amssymb}
\usepackage{amsmath}
%\usepackage{natbib}
\setcounter{tocdepth}{3}
\usepackage{graphicx}
%\graphicspath{ {Graficos/} }
\usepackage{subfigure}
\usepackage{gensymb}
\usepackage{authblk}
\usepackage{url}
\usepackage[utf8]{inputenc}

\usepackage[spanish]{babel}
\selectlanguage{spanish}
%\usepackage[style=authoryear]{biblatex}

%
%\newcommand{\keywords}[1]{\par\addvspace\baselineskip
%\noindent\keywordname\enspace\ignorespaces#1}
%\renewcommand\keywordname{Palabras Clave:}

\begin{document}

El presente trabajo buscó, mediante diferentes técnicas, aportar nueva evidencia empírica respecto de la organización del comercio internacional de bienes. Dado el objeto de estudio, la presentación del problema y los resultados se organizó en dos grandes bloques temáticos: En primer lugar el análisis de las relaciones bilaterales entre países, y en segundo lugar el rol de los países dados los productos que exportan. En ambos casos, el trabajo se estructuró a partir de un análisis exploratorio de datos, una definición metodológica del problema, y un análisis de los resultados obtenidos. 
En el capítulo dos, en donde se estudiaron las relaciones bilaterales, se observó el distinto rol que juegan los países en tanto productores y consumidores del mercado mundial. Por su parte, en el análisis de la serie 1948-2000, se pudo observar los fuertes cambios en las características topológicas de la red durante las décadas del 50' y 60', y el cambio en la centralidad del continente asiático. A su vez, en el análisis de la importancia de los países en forma individual, se observó un cambio en el rol de Japón, China, Corea de Sur y México, en dicho orden cronológico. 
En el capítulo tres, donde se estudiaron las relaciones comerciales a nivel producto, se observó en primer lugar las diferencias en la forma de comerciar de los países latinoamericano en tanto se trata del comercio regional, con el resto del mundo o con China específicamente. Luego, sobre la base de la propuesta metodológica de utilizar el modelo de \textit{Latent Dirichlet Allocation} en el presente objeto de estudio, se definió treinta dimensiones subyacentes del comercio internacional, que se utilizaron luego para caracterizar la evolución de la estructura productivo-exportadora de un conjunto amplio y diverso de países. Esta metodología nos permitió analizar la especificidad, y los cambios en la misma, del conjunto de países analizados, y dar con un panorama de la división internacional del trabajo. 
A su vez, se observaron los cambios técnicos ocurridos en la década del 70', y el fenómeno de la industrialización tardía en varios países de Asia, así como en México, Irlanda, España y Portugal. Además, la herramienta resultó lo suficientemente sensible como para poder caracterizar las diferencias en los distintos procesos de industrialización, y la diversificación productiva en Europa luego de los 70'. Respecto de América Latina, se observo la creciente importancia de la producción sojera en Argentina, Brasil y Uruguay, así como el cambio en Bolivia desde la producción de cobre a la producción gasífera, así como también la relevancia que las exportaciones de cobre tienen en Chile y Perú, y el petróleo en Venezuela. El análisis de la proyección de los nodos de países en el grafo bipartito, y su clustering posterior, confirma los resultados obtenidos mediante LDA, al mostrar una división, a partir de los años 70', entre países productores de materias primas, países industriales de baja complejidad y países industriales de alta complejidad. Se observa también allí, los cambios ocurridos en el continente asiático. Por su parte, el análisis de los clusters de la matriz de proximidad en el espacio de productos da cuenta de que los agrupamientos no solo son por ramas de las actividad, sino también por el grado de complejidad de los productos. En este análisis, sin embargo, no se logra la granularidad obtenida mediante LDA. 
El presente trabajo no se propuso indagar sobre las causas de la división internacional del trabajo, la industrialización tardía de ciertos países, ni el porqué del rol central que algunos países juegan en el grafo del comercio bilateral. El objetivo es realizar una serie de propuestas metodológicas para el análisis empírico del comercio mundial, en sus diferentes facetas, que puedan ser de utilidad para otros trabajos que indaguen respecto a dichas causas. Dado que se trata de generar nuevas representaciones de la información empírica disponible, de forma tal de echar luz sobre fenómenos largamente estudiados en la literatura, este trabajo hizo un uso intensivo de la visualización de la información como una herramienta propicia para objetivos propuestos. Por su parte, si la representación del capítulo tercero generaba la impresión de que se trata de un fenómeno en el cual elementos aislados, países, se interrelacionan; el capítulo cuarto resuelve dicha impresión dando la pauta de que el sistema económico funciona a nivel de sistema-mundo, y en donde los países son partes de una organización más general, que en la literatura se describe como la División Internacional del Trabajo. 
El presente trabajo deja como conclusión que la caracterización clásica de un mundo bipolar entre países industrializados, del \textit{centro} y países productores de materias primas, la \textit{periferia}, es insuficiente para dar cuenta de la complejidad que presenta la división geográfica actual de la producción. Por un lado países del \textit{centro} dan cuenta de un componente agrario importante en sus exportaciones, como los países nórdicos en Europa, Canadá, Australia y Nueva Zelanda. Por otro lado, países de la \textit{periferia} son los grandes productores industriales de la actualidad, en particular en el sudeste asiático, e incluso de productos de alta complejidad técnica, como los microprocesadores y transistores, entre otros. Este punto incluso se opone al análisis clásico de la Nueva División Internacional del trabajo, en dónde se planteaba que la relocalización productiva sólo sería para productos de baja complejidad técnica. Es importante destacar, sin embargo, que el presente análisis es también restringido respecto de su objeto de estudio. En particular, no se consideró en el presente trabajo la producción de servicios, lo cual incluye la multiplicidad de trabajos relacionados con la producción de software, contenidos digitales, tareas de diseño, investigación, etc. Por su parte, al considerar las exportaciones de un país, no se tiene en cuenta el origen de los capitales que producen dichos bienes. Estos dos elementos son indispensables para una caracterización de conjunto de la economía mundial, y para poder dar con las causas de los fenómenos que este trabajo simplemente se propuso graficar en sus expresiones empíricas. 
Las futuras líneas de investigación que este trabajo plantea, por lo tanto, es la utilización del herramental propuesto aquí como un insumo más en una explicación que busque dar cuenta de las causas de aquello que aquí se busco reflejar. 



\end{document}

