%\documentclass[../../tesis.tex]{subfiles}
\documentclass[class=article, crop=false]{standalone}
\usepackage[subpreambles=true]{standalone}
\usepackage{import}
\graphicspath{{images/}}


%
%
%
%
\usepackage{amssymb}
\usepackage{amsmath}
%\usepackage{natbib}
\setcounter{tocdepth}{3}
\usepackage{graphicx}
%\graphicspath{ {Graficos/} }
\usepackage{subfigure}
\usepackage{gensymb}
\usepackage{authblk}
\usepackage{url}
\usepackage[utf8]{inputenc}

\usepackage[spanish]{babel}
\selectlanguage{spanish}
%\usepackage[style=authoryear]{biblatex}


\newcommand{\keywords}[1]{\par\addvspace\baselineskip
\noindent\keywordname\enspace\ignorespaces#1}
%\renewcommand\keywordname{Palabras Clave:}

\begin{document}

El análisis del comercio internacional es una de las áreas de estudio más importantes de la investigación económica. Desde los comienzos de la economía política clásica constituye un tema de preocupación por el efecto del mismo en el desarrollo económico de los países \citep{ricardo1987principios}. Por su parte, el registro de la información referente al comercio entre países también se remonta en el tiempo.          

La cantidad de vínculos comerciales que se establecen entre entidades ubicadas en distintos países implica la imposibilidad de estudiar el fenómeno de forma directa, y plantea la necesidad de elaborar medidas de resumen que permitan apropiar la información subyacente al conjunto de los contratos comerciales existentes. 

Sin embargo, el análisis tradicional de la información generada carece de las herramientas necesarias para hacer frente a los grandes volúmenes de datos generados por el comercio internacional en la actualidad. Históricamente, los indicadores sintéticos del área se resumen en volumen y masa de dinero comerciada, partiendo del total mundial, hacia desagregaciones por región, país y sector económico en cuestión \cite{WTO2017}. De aquí se desprende la potencialidad del análisis que se basa en técnicas que logren captar en indicadores sucintos la complejidad del objeto de estudio.

A la vez que aumenta el volumen de información persistida respecto del comercio internacional, también se facilita el acceso a técnicas de análisis de datos que requieren un mayor poder de cómputo, y que por lo tanto eran impensadas como herramientas de estudio en épocas anteriores. En este sentido, surge la posibilidad de complementar el análisis tradicional del comercio mundial con técnicas de mayor riqueza en términos cuantitativos. 

El presente trabajo se propone utilizar las nuevas técnicas que provee el análisis de grafos para caracterizar el comercio internacional. Su modelización como una red compleja permite la construcción de medidas de resumen que, sin abandonar una mirada holística de la problemática, logren dar cuenta de una mayor complejidad que las métricas tradicionales de dicha área temática. 

Una de las posibles formas en que se puede desarrollar dicho análisis es considerar a cada país como la unidad fundamental a partir de la cual se forma una red basada en sus relaciones comerciales con otros países-nodos. Este tipo de análisis permite tener una mirada general de la posición de los países en el comercio internacional, caracterizar su importancia relativa, así como también inferir propiedades estructurales del comercio internacional en su conjunto. En el capítulo dos se desarrolla el estudio siguiendo esta premisa. 

Sin embargo, existe aún un nivel mayor de complejidad que puede ser analizado. Es posible descomponer el comercio bilateral entre dos países en los múltiples contratos comerciales individuales entre entidades de los diferentes países. Si bien ese nivel de desagregación es el máximo posible en el objeto de estudio, dicha información no se encuentra disponible al público. Lo que sí es accesible al público es la información de los volúmenes comerciados de cada tipo de producto entre cada par de países, siguiendo un nomenclador estandarizado como el 'Standard International Trade Classification' \citep{un2006standard} o el 'Harmonized System' \citep{weerth2008basic}. 
Este nivel de desagregación implica que si se continúa considerando a los países como la unidad fundamental de análisis, entre cada par de nodos puede existir una multiplicidad de aristas, potencialmente una por cada producto definido por el nomenclador. Esto implica la construcción de un grafo bipartito entre países y productos. Dicho modelo se analiza en el capítulo \ref{sec:desagregado}. 

Por su parte, los nomencladores mencionados tienen la propiedad de ser sistemas jerárquicos en los cuales un mismo producto final pertenece a una serie de categorías intermedias, la tabla  \ref{table:ejemplo-hs} muestra un ejemplo de los distintos niveles de desagregación que presenta el arroz. Esta estructura de datos permite realizar un análisis exploratorio de la información a través de los diferentes niveles pero a su vez plantea la cuestión de cual es la mejor estructura jerárquica para el análisis de la información. Frente a esto, la multiplicidad de nomencladores, tanto oficiales como realizados ad-hoc para análisis específicos \citep{molinari2016especializacion} plantean que no existe un único sistema de estandarización de la información. De aquí surge la posibilidad de utilizar técnicas de reducción de dimensionalidad que utilicen la información disponible para agrupar los productos según grupos de pertenencia. En el tercer capítulo de este trabajo también se avanza en esta dirección proponiendo la utilización del modelo de Latent Dirichlet Allocation \citep{blei2003latent} para explotar la información disponible en la elaboración de un sistema de clustering difuso. 

\begin{table}[]
	\begin{tabular}{ll}
		\textbf{Nivel}     & \textbf{Definición}                                          \\
		Sección II         & Productos Vegetales                                          \\
		Capítulo 10        & Cereales                                                     \\
		Partida 10.06      & Arroz                                                        \\
		Subpartida 1006.30 & Arroz semiblanqueado o blanqueado, incluso pulido o glaseado
	\end{tabular}
	\caption{Ejemplo de desagregación para el Sistema Harmonizado}
	\label{table:ejemplo-hs}
\end{table}


A su vez, tomando el concepto de Ventajas Comparativas Reveladas \citep{balassa1965trade} se puede calcular un grafo bipartito entre países y productos. La proyección al grafo de países permite utilizar técnicas de detección de comunidades definidas a partir de la semejanza de su estructura exportadora. Finalmente, también en el capítulo tercero, se toma la metodología propuesta por \cite{Hidalgo2009} para caracterizar el espacio de productos a partir del concepto de \textit{similarity}. Esto permite tomar la matriz de similitud resultante como una matriz de adyacencia y construir con la misma un grafo. 



\section{Fuentes de información}


La modelización del comercio internacional como un grafo requiere de los datos del flujo anual de comercio bilateral para el total de las mercancías, para la mayor cantidad posible de países, idealmente todos ellos. Dado que los datos del comercio bilateral son realizados por cada país involucrado, es necesaria una base de datos en la cual se haya recolectado la información provista por cada país, y que a su vez haya sido consolidada frente a las posibles, y de hecho abundantes, contradicciones que se presentan entre los reportes oficiales de los países involucrados. En particular, dado que toda importación para un país es una exportación de otro, y que el registro de los datos se realiza de forma nacional, existe una duplicación formal de la información, que no siempre es coincidente. Por esto, se recurrió a una base previamente consistida por un organismo internacional oficial, la Organización Mundial del Comercio (de aquí en más OMC)\footnote{https://comtrade.un.org/}, y la API que dicho organismo provee para descarga de datos. La información analizada  en el capítulo \ref{sec:agregado} para el período 1996-2016, proviene de dicho organismo.

Los datos utilizados para el análisis de largo plazo, entre los años 1948 y 2000 proviene del trabajo realizado por \cite{Gleditsch2002}, que también es utilizado en otros trabajos de modelización del comercio internacional mediante redes complejas \cite{Fagiolo2010}.

En el capítulo \ref{sec:desagregado}, donde se analiza el comercio a nivel producto, requiera de las exportaciones de cada país desagregadas por un nomenclador. Para ello, se utilizó la información provista por el \textit{\cite{star}}, la cual se basa en la información de comtrade, consistiendo los datos faltantes a partir de los flujos en espejo, y los distintos nomencladores utilizados en el tiempo. Esta base nos permite realizar un análisis histórico, y es la utilizada por la literatura especializada \citep{Hidalgo2007, Hidalgo2009, Hidalgo2009a}. Sin embargo, dado que la normalización implica un cierta pérdida de información, para la sección \ref{sec:espacio_productos}, se decidió reconstruir la base directamente a partir de los datos de \textit{comtrade}.

\section{Software utilizado}

Para el presente trabajo se utilizó el lenguaje de programación estadística \texttt{\textbf{R}} \citep{RCoreTeam2017}, junto con sendas extensiones del mismo. Aquellas que resulta importante mencionar son \texttt{igraph}\citep{Csardi2006} para la construcción de los grafos y las medidas de resumen de los mismos; \texttt{tidyverse}\citep{Wickham2017} para la manipulación de la información y elaboración de los gráficos. Para esto último, también se hizo uso de \textit{ggridges} \citep{Wilke2017}, \textit{ggthemes} \citep{Arnold2017}, \textit{RColorBrewer} \citep{Neuwirth2014} y \textit{ggrepel} \citep{Slowikowski2017} como complementos de \texttt{ggplot}. Por su parte, los códigos de los países provienen de \citep{Arel-Bundock2017}.

Para el análisis del capítulo \ref{sec:desagregado} también se utilizó el lenguaje \texttt{\textbf{Python}} \citep{CS-R9526}, y específicamente la librería \textit{scikit-learn} \citep{scikit-learn} para la estimación de los modelos.
A su vez, para esta sección también se elaboraron dos tableros dinámicos para el análisis de resultados, sobre la base de la librería \textit{shiny} \citep{Chang2018}, y se hostearon en \url{www.shinyapps.io/} \citep{Allaire2019}


%\bibliographystyle{unsrt}
%\bibliography{bibliografia}
%
\end{document}

