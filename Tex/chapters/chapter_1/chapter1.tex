%\documentclass[../../tesis.tex]{subfiles}
\documentclass[class=article, crop=false]{standalone}
\usepackage[subpreambles=true]{standalone}
\usepackage{import}
\graphicspath{{images/}}


%
%
%
%
\usepackage{amssymb}
\usepackage{amsmath}
%\usepackage{natbib}
\setcounter{tocdepth}{3}
\usepackage{graphicx}
%\graphicspath{ {Graficos/} }
\usepackage{subfigure}
\usepackage{gensymb}
\usepackage{authblk}
\usepackage{url}
\usepackage[utf8]{inputenc}

\usepackage[spanish]{babel}
\selectlanguage{spanish}
%\usepackage[style=authoryear]{biblatex}


\newcommand{\keywords}[1]{\par\addvspace\baselineskip
\noindent\keywordname\enspace\ignorespaces#1}
%\renewcommand\keywordname{Palabras Clave:}

\begin{document}

El análisis del comercio internacional es una de las áreas de estudio más importantes de la investigación económica. Desde los comienzos de la economía política clásica constituye un tema de preocupación por el efecto del mismo en el desarrollo económico de los países\cite{ricardo1987principios}. Por su parte, el registro de la información referente al comercio entre países también se remonta en el tiempo.          

La cantidad de vínculos comerciales que se establecen entre entidades ubicadas en distintos países implica la imposibilidad de estudiar el fenómeno de forma directa, y plantea la necesidad de elaborar medidas de resumen que permitan apropiar la información subyacente al conjunto de los contratos comerciales existentes. 

Sin embargo, el análisis tradicional de la información generada carece de las herramientas necesarias para hacer frente a los grandes volúmenes de datos generados por el comercio internacional en la actualidad. Históricamente, los indicadores sintéticos del área se resumen en volumen y masa de dinero comerciada, partiendo del total mundial, hacia desagregaciones por región, país y sector económico en cuestión \cite{WTO2017}. De aquí se desprende la potencialidad del análisis que se basa en técnicas que logren captar en indicadores sucintos la complejidad del objeto de estudio.

A la vez que aumenta el volumen de información persistida respecto del comercio internacional, también se facilita el acceso a técnicas de análisis de datos que requieren un mayor poder de cómputo, y que por lo tanto eran impensadas como herramientas de estudio en épocas anteriores. En este sentido, surge la posibilidad de complementar el análisis tradicional del comercio mundial con técnicas de mayor riqueza en términos cuantitativos. 

El presente trabajo se propone utilizar las nuevas técnicas que provee el análisis de grafos para caracterizar el comercio internacional. Su modelización como una red compleja permite la construcción de medidas de resumen que, sin abandonar una mirada holística de la problemática, logren dar cuenta de una mayor complejidad que las métricas tradicionales de dicha área temática. 

Una de las posibles formas en que se puede desarrollar dicho análisis es considerar a cada país como la unidad fundamental a partir de la cual se forma una red basada en sus relaciones comerciales con otros países-nodos. Este tipo de análisis permite tener una mirada general de la posición de los países en el comercio internacional, caracterizar su importancia relativa, así como también inferir propiedades estructurales del comercio internacional en su conjunto. En el capítulo dos se desarrolla el estudio siguiendo esta premisa. 

Sin embargo, existe aún un nivel mayor de complejidad que puede ser analizado. Es posible descomponer el comercio bilateral entre dos países en los múltiples contratos comerciales individuales entre entidades de los diferentes países. Si bien ese nivel de desagregación es el máximo posible en el objeto de estudio, dicha información no se encuentra disponible al público. Lo que sí es accesible al público es la información de los volúmenes comerciados de cada tipo de producto entre cada par de países, siguiendo un nomenclador estandarizado como el 'Standard International Trade Classification' \citep{united2006standard} o el 'Harmonized System' \citep{weerth2008basic}. 
Este nivel de desagregación implica que si se continúa considerando a los países como la unidad fundamental de análisis, entre cada par de nodos puede existir una multiplicidad de aristas, potencialmente una por cada producto definido por el nomenclador. Esto implica la construcción de un grafo bipartito entre países y productos. Dicho modelo se analiza en el capítulo tercero de esta tesis. 

Por su parte, nos nomencladores mencionados tienen la propiedad de ser sistemas jerárquicos en los cuales un mismo producto final pertenece a una serie de categorías intermedias, la tabla  \ref{table:ejemplo-hs} muestra un ejemplo de los distintos niveles de desagregación que presenta el arroz. Esta estructura de datos permite realizar un análisis exploratorio de la información a través de los diferentes niveles pero a su vez plantea la cuestión de cual es la mejor estructura jerárquica para el análisis de la información. Frente a esto, la multiplicidad de nomencladores, tanto oficiales como realizados ad-hoc para análisis específicos \citep{molinari2016especializacion} plantean que no existe un único sistema de estandarización de la información. De aquí surge la posibilidad de utilizar técnicas de reducción de dimensionalidad que utilicen la información disponible para agrupar los productos según grupos de pertenencia. En el tercer capítulo de este trabajo también se avanza en esta dirección proponiendo la utilización del modelo de Latent Dirichlet Allocation \citep{blei2003latent} para explotar la información disponible en la elaboración de un sistema de clustering difuso. 


\begin{table}[]
	\begin{tabular}{ll}
		\textbf{Nivel}     & \textbf{Definición}                                          \\
		Sección II         & Productos Vegetales                                          \\
		Capítulo 10        & Cereales                                                     \\
		Partida 10.06      & Arroz                                                        \\
		Subpartida 1006.30 & Arroz semiblanqueado o blanqueado, incluso pulido o glaseado
	\end{tabular}
	\caption{Ejemplo de desagregación para el Sistema Harmonizado}
	\label{table:ejemplo-hs}
\end{table}


\pendiente{ver que pasa con grafos bipartitos}



%\bibliographystyle{unsrt}
%\bibliography{bibliografia}
%
\end{document}

