%\documentclass[../../tesis.tex]{subfiles}
\documentclass[class=article, crop=false]{standalone}
\usepackage[subpreambles=true]{standalone}
\usepackage{import}
\graphicspath{{images/}}


%
%
%
%
\usepackage{amssymb}
\usepackage{amsmath}
%\usepackage{natbib}
\setcounter{tocdepth}{3}
\usepackage{graphicx}
%\graphicspath{ {Graficos/} }
\usepackage{subfigure}
\usepackage{gensymb}
\usepackage{authblk}
\usepackage{url}
\usepackage[utf8]{inputenc}

\usepackage[spanish]{babel}
\selectlanguage{spanish}
%\usepackage[style=authoryear]{biblatex}


\newcommand{\keywords}[1]{\par\addvspace\baselineskip
\noindent\keywordname\enspace\ignorespaces#1}
%\renewcommand\keywordname{Palabras Clave:}

\begin{document}

El análisis del comercio internacional es una de las áreas de estudio más importantes de la investigación económica. Desde los comienzos de la economía política clásica constituye un tema de preocupación por sus fuertes implicancias \cite{ricardo1987principios}. Por su parte, el registro de la información referente al comercio entre países también se remonta en el tiempo.          

Sin embargo, el análisis tradicional de la información generada carece de las herramientas necesarias para hacer frente a los grandes volúmenes de datos generados por el comercio internacional en la actualidad. Históricamente, los indicadores sintéticos del área se resumen en volumen y masa de dinero comerciada, partiendo del total mundial, hacia desagregaciones por región, país y sector económico en cuestión \cite{WTO2017}.                

A la vez que aumenta el volumen de información persistida respecto del comercio internacional, también se facilita el acceso a técnicas de análisis de datos que requieren un mayor poder de cómputo, y que por lo tanto eran impensadas como herramientas de estudio en épocas anteriores. En este sentido, surge la posibilidad de complementar el análisis tradicional del comercio mundial con técnicas de mayor riqueza en términos cuantitativos. 

El presente trabajo se propone utilizar las nuevas técnicas que provee el análisis de grafos para caracterizar el comercio internacional. Su modelización como una red compleja permite la construcción de medidas de resumen que, sin abandonar una mirada holística de la problemática, logren dar cuenta de una mayor complejidad que las métricas tradicionales de dicha área temática. 

\pendiente{datos agregados, desagregados}



%\bibliographystyle{unsrt}
%\bibliography{bibliografia}
%
\end{document}

