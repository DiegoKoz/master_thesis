%\begin{center}
%\large \bf \runtitulo
%\end{center}
%\vspace{1cm}
\chapter*{\runtitulo}

\noindent

El estudio del comercio internacional es una de las áreas de investigación clásicas de las ciencias económicas. Su caracterización empírica, a la vez que se remonta en el tiempo, constituye hoy en día un espacio de aplicación para nuevas técnicas, basadas en el uso intensivo de datos. El aumento en la disponibilidad de información curada para el conjunto de los países permite explorar nuevas técnicas de análisis. A su vez, las múltiples dimensiones del problema impiden que una única técnica de análisis permita capturar la totalidad del fenómeno. El presente trabajo se propone realizar diversas propuestas metodológicas para el estudio del comercio internacional, de forma tal que en su conjunto permitan obtener una visión novedosa de los datos. Para ello, se utilizan técnicas ampliamente difundidas en otros campos, como el análisis de grafos y los modelos generativos bayesianos, y se considera el comercio agregado entre países, así como la composición de las canastas exportadoras de los mismos. 

\bigskip

\noindent\textbf{Palabras claves:} Comercio Internacional, Grafos, Ciencia de Datos, Dirección Latente de Dirichlet, Aprendizaje no Supervisado.