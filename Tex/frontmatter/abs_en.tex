%\begin{center}
%\large \bf \runtitle
%\end{center}
%\vspace{1cm}
\chapter*{\runtitle}

\noindent 

The study of international trade is one of the classical research areas of the economic thought. Its empirical characterization, at the same time that it goes back in time, is today an application space for new techniques, based on the intensive use of data. The increase in the availability of information curated for all countries allows exploring new analysis techniques. In turn, the multiple dimensions of the problem prevent a single analysis technique from capturing the entire phenomenon. The present work proposes to carry out various methodological proposals for the study of international trade, in such a way that together they allow obtaining a novel vision of the data. To do this, techniques widely used in other fields are tested on this particular subject, such as graph analysis and bayesian generative models.  Aggregate trade between countries as well as the composition of the export baskets are considered.

\bigskip

\noindent\textbf{Keywords:} World Trade, Graphs, Data Science, Latent Dirichlet Allocation, Unsupervised Learning.