%\documentclass[11pt,a4paper,twoside]{frontmatter/tesis}
% SI NO PENSAS IMPRIMIRLO EN FORMATO LIBRO PODES USAR
\documentclass[11pt,a4paper]{tesis}

% subfiles structure
%\usepackage{subfiles}
\usepackage[subpreambles=true]{standalone}
\usepackage{import}

%other packages
\usepackage{graphicx}
\usepackage[utf8]{inputenc}
\usepackage[spanish]{babel}
\usepackage[left=3cm,right=3cm,bottom=3.5cm,top=3.5cm]{geometry}

%\bibliographystyle{apalike}
\usepackage[authoryear,round]{natbib}

\bibliographystyle{myplainnat}


\renewcommand{\betweenauthors}{&}
%\usepackage[style=authoryear]{biblatex}

%general adjustments
%\setcounter{tocdepth}{3}
%\graphicspath{ {images/}{../../images}] }


%librerias chapter 1
\usepackage{amssymb}
\usepackage{amsmath}
%\usepackage{natbib}
\usepackage{subfigure}
\usepackage{gensymb}
\usepackage{authblk}
\usepackage{url}


\usepackage{hyperref}
\usepackage{cleveref}
\usepackage{nameref}

\usepackage[subfigure]{tocloft}


%para hacer comentarios
\usepackage{xargs}                      % Use more than one optional parameter in a new commands
\usepackage[pdftex,dvipsnames]{xcolor}  % Coloured text etc.
\usepackage[colorinlistoftodos,prependcaption,textsize=tiny]{todonotes}
\newcommandx{\unsure}[2][1=]{\todo[linecolor=red,backgroundcolor=red!25,bordercolor=red,#1]{#2}}
\newcommandx{\change}[2][1=]{\todo[linecolor=blue,backgroundcolor=blue!25,bordercolor=blue,#1]{#2}}
\newcommandx{\pendiente}[2][1=]{\todo[linecolor=OliveGreen,backgroundcolor=OliveGreen!25,bordercolor=OliveGreen,#1]{#2}}
\newcommandx{\info}[2][1=]{\todo[linecolor=Plum,backgroundcolor=Plum!25,bordercolor=Plum,#1]{#2}}
\newcommandx{\thiswillnotshow}[2][1=]{\todo[disable,#1]{#2}}


%%%%%%%%%%%%% Titulos

%%%%%%%%%%%% paragraph con titulo en negrita
\newcommand{\partitle}[1]{%
	\addvspace{0.2cm}% perhaps, as Barbara Beeton suggests
	\textbf{#1}
	\par
	\hspace{0.5cm}
}


%%%%%%%%% \subsubsection en negrita e italica
\usepackage{titlesec}

\titleformat{\subsubsection}
{\normalfont\fontfamily{phv}\fontsize{10}{17}\bfseries\itshape}{\thesubsection}{1em}{}



\begin{document}

%%%% CARATULA

\def\autor{Diego Kozlowski}
\def\tituloTesis{Técnicas de Ciencia de Datos aplicadas al comercio internacional \change{provisorio}}
\def\runtitulo{Técnicas de Ciencia de Datos aplicadas al comercio internacional \change{provisorio}}
\def\runtitle{Data Science Techniques applied to World Trade \change{provisorio}}
\def\director{Dra. Viktoriya Semeshenko}
%\def\codirector{}
\def\lugar{Buenos Aires, 2019}
\newcommand{\HRule}{\rule{\linewidth}{0.2mm}}
%
\thispagestyle{empty}

\begin{center}\leavevmode

\vspace{-2cm}

\begin{tabular}{l}
\includegraphics[width=2.6cm]{frontmatter/logofcen.pdf}
\end{tabular}


{\large \sc Universidad de Buenos Aires

Facultad de Ciencias Exactas y Naturales

Maestría en Explotación de Datos y Descubrimiento del Conocimiento}

\vspace{2.0cm}

%\vspace{3.0cm}
%{
%\Large \color{red}
%\begin{tabular}{|p{2cm}cp{2cm}|}
%\hline
%& Pre-Final Version: \today &\\
%\hline
%\end{tabular}
%}
%\vspace{2.5cm}

\begin{huge}
\textbf{\tituloTesis}
\end{huge}

\vspace{2cm}

{\large Tesis presentada para optar al Título de Magister de la Universidad de Buenos Aires en Explotación de Datos y Descubrimiento del Conocimiento}

\vspace{2cm}

{\Large \autor}

\end{center}

\vfill

{\large

{Directora: \director}

\vspace{1cm}

%{Codirector: \codirector}


\lugar
\vspace{1cm}

\fechapre

\vspace{1cm}
\fechadef

}

\newpage\thispagestyle{empty}


%%%% ABSTRACTS, AGRADECIMIENTOS Y DEDICATORIA
\frontmatter
\pagestyle{empty}
%\begin{center}
%\large \bf \runtitulo
%\end{center}
%\vspace{1cm}
\chapter*{\runtitulo}

\noindent

El estudio del comercio internacional es una de las áreas de investigación clásicas de las ciencias económicas. Su caracterización empírica, a la vez que se remonta en el tiempo, constituye hoy en día un espacio de aplicación para nuevas técnicas, basadas en el uso intensivo de datos. El aumento en la disponibilidad de información curada para el conjunto de los países permite explorar nuevas técnicas de análisis. A su vez, las múltiples dimensiones del problema impiden que una única técnica de análisis permita capturar la totalidad del fenómeno. El presente trabajo se propone realizar diversas propuestas metodológicas para el estudio del comercio internacional, de forma tal que en su conjunto permitan obtener una visión novedosa de los datos. Para ello, se utilizan técnicas ampliamente difundidas en otros campos, como el análisis de grafos y los modelos generativos bayesianos, y se considera el comercio agregado entre países, así como la composición de las canastas exportadoras de los mismos. 

\bigskip

\noindent\textbf{Palabras claves:} Comercio Internacional, Grafos, Ciencia de Datos, Dirección Latente de Dirichlet, Aprendizaje no Supervisado.

\cleardoublepage
%\begin{center}
%\large \bf \runtitle
%\end{center}
%\vspace{1cm}
\chapter*{\runtitle}

\noindent 

The study of international trade is one of the classical research areas of the economic thought. Its empirical characterization, at the same time that it goes back in time, is today an application space for new techniques, based on the intensive use of data. The increase in the availability of information curated for all countries allows exploring new analysis techniques. In turn, the multiple dimensions of the problem prevent a single analysis technique from capturing the entire phenomenon. The present work proposes to carry out various methodological proposals for the study of international trade, in such a way that together they allow obtaining a novel vision of the data. To do this, techniques widely used in other fields are tested on this particular subject, such as graph analysis and bayesian generative models.  Aggregate trade between countries as well as the composition of the export baskets are considered.

\bigskip

\noindent\textbf{Keywords:} World Trade, Graphs, Data Science, Latent Dirichlet Allocation, Unsupervised Learning. % OPCIONAL: comentar si no se quiere

\cleardoublepage
\chapter*{Agradecimientos}

\noindent 

A mi familia, mi compañera Natsumi, y mis amigos por el apoyo constante y ánimos para avanzar en este trabajo. 

\bigskip

A Leonardo Córdoba, por sus valiosos comentarios y discusiones sobre versiones anteriores de este trabajo.

\bigskip

A los organizadores del Young Schollar Initiative y LatinR y ECON por darme el lugar para exponer versiones anteriores de este trabajo y recibir valiosos comentarios.

\bigskip

A Viktoriya Semeshenko por aceptar dirigir esta tesis. A Andrea Molinari, quién junto a Viktoriya a realizado valiosos comentarios y sugerencias, y con quien hemos discutido en profundidad partes de este trabajo. Las discusiones con ambas han enriquecido enormemente este trabajo.

\bigskip

A Marcelo Soria, por dirigir el trabajo de especialización con el que comenzó este proyecto, y su excepcional apoyo a lo largo de todo el proceso. 


 % OPCIONAL: comentar si no se quiere

%\cleardoublepage
%\input{dedicatoria.tex}  % OPCIONAL: comentar si no se quiere

\cleardoublepage
\tableofcontents

\mainmatter
\pagestyle{headings}

%%%% ACA VA EL CONTENIDO DE LA TESIS

\chapter{introducción}\label{sec:introduccion}

\import{chapters/}{chapter_1/chapter1}

\chapter{Comercio a nivel países}\label{sec:agregado}

\import{chapters/}{chapter_2/chapter2}

%\section{Introducción}
%\info{inclyue estado del arte}
%
%\section{Análisis exploratorio de datos}
%
%\section{Metodología}
%
%\section{Resultados}
%
%\subsection{El comercio internacional en el 2016 a la luz del análisis de redes}
%
%

%---------------------------

\chapter{Comercio a nivel producto}\label{sec:desagregado}

\import{chapters/}{chapter_3/chapter3}



\chapter{Conclusiones}\label{sec:conclusiones}

\appendix \label{append}

\import{chapters/}{Anexo/anexo}



%%%% BIBLIOGRAFIA
\backmatter
%\bibliographystyle{plainnat}
\bibliography{backmatter/bibliografia}

%\bibliography{bibliografia}

\end{document}
